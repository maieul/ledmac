\documentclass[a3paper]{book}


\usepackage[noeledsec,nofamiliar,series={A}]{reledmac}

\firstlinenum{0}
\linenumincrement{1}

\Xnumberonlyfirstinline[A]
\Xinplaceofnumber[A]{26mm}
\Xtwolines{f.}
\Xtwolinesbutnotmore
\Xhangindent{2.6cm}
\Xboxlinenum{2.6cm}
\Xboxstartlinenum{1.3em}
\Xboxendlinenum{0.5em}
\Xafternumber{0em}
\Xsublinesep{}



\Xendinplaceofnumber[A]{26mm}
\Xendtwolines{f.}
\Xendtwolinesbutnotmore
\Xendhangindent{2.6cm}
\Xendboxlinenum{2.6cm}
\Xendboxstartlinenum{1.3em}
\Xendboxendlinenum{0.5em}
\Xendafternumber{0em}
\Xendsublinesep{}

\sublinenumberstyle{alph}

\newcommand{\Anote}[1]{%
  \Afootnote{#1}%
  \Aendnote{#1}%
}
\renewcommand{\printnpnum}[1]{}%Just for this example, we don't need page number in endnote.

\begin{document}\setlength{\parindent}{0pt}

\beginnumbering

\leftskip=2.6cm
\rightskip=1cm

\pstart
Head: Subline number position
\pend
\vspace{0.75em}
\pstart\leftskip=0cm\edtext{\raisebox{-.25\baselineskip}[\ht\strutbox]{\parbox{2.6cm} {\small\textit{Hand-}}}This text was written using an old school typewriter. Only line\linebreak\raisebox{-.25\baselineskip}[\ht\strutbox]{\parbox{2.6cm} {\small\textit{written}}}numbers of this text are printed in the margin of the main sec-\linebreak
\raisebox{-.25\baselineskip}[\ht\strutbox]{\parbox{2.6cm} {\small\textit{Addition}}}}{\lemma{\textit{Handwritten Addition}}\linenum{||1|||1}\Anote{added manually}}tion of this page. Therefore, subline numbers are only printed in\linebreak\parbox{2.6cm}{{\ }}the footnotes.
\pend
\vspace{0.4em}
\pstart There is no subline seperator, only a \edtext{symbol}{\lemma{symbol}\Anote{typewritten correction of: sombyl}} attached without spacing to the full line number ("1a", e.g.). This symbol should not alter the position of the (sub-)line number, that is the digits should remain exactly above (and/or below) the digits of \edtext{footnotes}{\lemma{footnotes}\Anote{handwritten correction of: rootnotes}} refering to full lines (contrary to what it looks now). \edtext{That means that a subline number in the footnotes should extend into the space between the numbers and the lemma, similar to what a range symbol like "f." or a dash followed by an endnumber do, when Xboxstartlinenum and Xboxendlinenum are enabled}{\lemma{That\ldots enabled}\Anote{A long note}}.
\pend
\vspace{0.4em}
\pstart\rightskip=0cm There is yet another paragraph where alterations occur, \edtext{too}{\lemma{too}\Anote{typewritten correction of: two}}, so \edtext{\raisebox{.5\baselineskip}[\ht\strutbox]{\parbox{1cm} {\small{[?]}}}}{\lemma{[?]}\linenum{||1|||1}\Anote{unidentified handwritten addition}}\linebreak there will be footnotes in full lines and sublines for two-digit-\parbox{1cm}{{\ }}\linebreak lines as well.
\pend

\endnumbering

\vspace{1cm}
\textbf{Endnotes, just for testing}

\doendnotes{A}

\end{document}
