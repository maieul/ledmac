% !TEX encoding = UTF-8 Unicode
% !TEX program = xelatex
% !TEX spellcheck = it_IT
%---------------------------------------------------------------------------------------
% PACKAGES
%---------------------------------------------------------------------------------------
\documentclass[11pt,a4paper]{book}
\usepackage[libertine={Ligatures=TeX,Numbers=OldStyle}]{libertineotf}
\usepackage{ledmac,ledpar}
%\usepackage{ledpar} %Necessario per \emph in ledsidenote, ma confligge con nosep/nonum

%%%%%%%%%%%%%%%%%%%%%%%%%%%%%%%%%%%%%%%%%%%%%%%%%%%%%%%%%%%%%%%%%%%%%%%%
%----------------------------------------------------------------------%
%-------------INIZIO EDIZIONE CRITICA (LEDMAC)-------------------------%
%----------------------------------------------------------------------%
%%%%%%%%%%%%%%%%%%%%%%%%%%%%%%%%%%%%%%%%%%%%%%%%%%%%%%%%%%%%%%%%%%%%%%%%

\lineation{page}                %% numerazione per pagina
\linenummargin{inner}   %% Margine dei numeri di linea
\sidenotemargin{outer}  %% Margine dei marginalia

\renewcommand*{\notenumfont}{\footnotesize} %%Font numeri linea note
\newcommand*{\notetextfont}{\footnotesize}      %%Font testo apparato

%Formato marginalia
\renewcommand*{\ledlsnotefontsetup}{\raggedleft\it\footnotesize}    % left
\renewcommand*{\ledrsnotefontsetup}{\raggedright\it\footnotesize}   % right


%Azzeriamo lo spazio residuo, una volta tolto il separatore lemma
\inplaceoflemmaseparator[A]{0em}    %NOT IN 0.18
\inplaceoflemmaseparator[B]{0em}    %NOT IN 0.18
\inplaceoflemmaseparator[C]{.5em}   %NOT IN 0.18

%%% SPAZIO LIBERO SOPRA LE RIGHE SEPARATRICI
\addtolength{\skip\Afootins}{2em plus.4em minus.4em}

% SPAZIO BIANCO FRA TESTO ED APPARATO (def. = 5mm)
\setlength{\skip\Afootins}{2em plus.4em minus.4em}

% SPAZIO BIANCO FRA NOTE D'APPARATO
\afternote[A]{1em plus.4em minus.4em}
\afternote[B]{1em plus.4em minus.4em}
\afternote[C]{1em plus.4em minus.4em}

\footparagraph{A}
\footparagraph{B}
\footparagraph{C}

%% DEFINIZIONE NEWPARA
\newcounter{para}[chapter]\setcounter{para}{0}
    \newcommand{\newpara}{%
    \refstepcounter{para}%
    \noindent\llap{\thepstart}}
    \newcommand{\oldpara}[1]{%
    \noindent\llap{\ref{#1}}}

%%%%%%%%%%%%%%%%%%%%%%%%%%%%%%%%%%%%%%%%%%%%%%%%%%%%%%%%%%%%%%%%%%%%%%%%
% ---------------------------------------------------------------------%
% ----------------------FINE EDIZIONE CRITICA--------------------------%
% ---------------------------------------------------------------------%
%%%%%%%%%%%%%%%%%%%%%%%%%%%%%%%%%%%%%%%%%%%%%%%%%%%%%%%%%%%%%%%%%%%%%%%%

\begin{document}
\begin{pages}
\begin{Rightside}
\beginnumbering

 \pstart \edtext{s}{\Afootnote[nosep]{s}}· “ἐσπούδακας, ὦ Φαῖδρε, ὅτι σου τῶν παιδικῶν ἐπελαβόμην, ἐρεσχηλῶν σε” (Phaedr. 236b5-6). ἡ δὲ λέξις ὡς ἐπὶ τὸ πολὺ ἐπὶ τῶν ἀσελγῶς\pend

\numberpstartfalse

\endnumbering
\end{Rightside}
\begin{Leftside}
\beginnumbering

 \pstart \edtext{s}{\Afootnote{s}}· “ἐσπούδακας, ὦ Φαῖδρε, ὅτι σου τῶν παιδικῶν ἐπελαβόμην, ἐρεσχηλῶν σε” (Phaedr. 236b5-6). ἡ δὲ λέξις ὡς ἐπὶ τὸ πολὺ ἐπὶ τῶν ἀσελγῶς\pend

\numberpstartfalse

\endnumbering
\end{Leftside}
\begin{Leftside}
\beginnumbering

 \pstart \edtext{s}{\edtext{a}{\Afootnote{s}}\Afootnote{s}}· “ἐσπούδακας, ὦ Φαῖδρε, ὅτι σου τῶν παιδικῶν ἐπελαβόμην, ἐρεσχηλῶν σε” (Phaedr. 236b5-6). ἡ δὲ λέξις ὡς ἐπὶ τὸ πολὺ ἐπὶ τῶν ἀσελγῶς\pend

\numberpstartfalse

\endnumbering
\end{Leftside}
\Pages
\end{pages}
\beginnumbering

 \pstart \edtext{s}{\edtext{a}{\Afootnote{s}}\Afootnote{s}}· “ἐσπούδακας, ὦ Φαῖδρε, ὅτι σου τῶν παιδικῶν ἐπελαβόμην, ἐρεσχηλῶν σε” (Phaedr. 236b5-6). ἡ δὲ λέξις ὡς ἐπὶ τὸ πολὺ ἐπὶ τῶν ἀσελγῶς\pend
\numberpstartfalse

\endnumbering
\end{document}