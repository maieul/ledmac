\documentclass{book}
\usepackage{geometry}
\usepackage[utf8]{inputenc}
\usepackage[T1]{fontenc}
\usepackage[german]{babel}
\usepackage{lmodern}
\usepackage{eledmac} 

\firstlinenum{1}  \linenumincrement{1}%

\begin{document}
\beginnumbering
\pstart
\begin{edtabularl}
linea prima & linea \edtext{prima}{\Afootnote{primma}} \\
linea secunda & linea \edtext{seconda}{\Afootnote[nosep]{pro secunda}} \\
linea tertia & linea \edtext{tertia}{\Afootnote[nonum]{tercia}} \\
linea quarta & linea quarta 
\end{edtabularl}
\pend
\endnumbering
\vspace{5ex}
\beginnumbering
\pstart
linea prima  linea \edtext{prima}{\Afootnote{primma}} 
linea secunda  linea \edtext{seconda}{\Afootnote[nosep]{pro secunda}} 
linea tertia  linea \edtext{tertia}{\Afootnote[nonum]{tercia}} 
linea quarta  linea quarta 
\pend
\endnumbering
\end{document}