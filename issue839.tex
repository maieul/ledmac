
\documentclass{article}
\usepackage[english,french, bidi=basic]{babel}
\babeltags{ar = arabic}
\babelfont{rm}{Crimson}
\babelfont[*arabic]{rm}{Amiri}
\usepackage{arabluatex}
\usepackage[series={A,B}]{reledmac}
\Xarrangement{paragraph}
\Xbhookgroup[A]{\pardir TRT\textdir TRT}
\Xlemmaseparator[A]{[}
\Xafterlemmaseparator[A]{0.5em}
\Xbeforelemmaseparator[A]{0.25em}
\Xwrapcontent[A]{\txarb}
% I think the following is needed to print subsequent line numbers in
% LTR:
\Xbhooknote[A]{\textdir TRT}
\firstlinenum{1} \linenumincrement{1} \lineation{page}

\renewcommand{\Afootnoterule}{%
\hfill\noindent\rule[1mm]{.4\textwidth}{.15mm}}

\begin{document}
\beginnumbering
\setlinenum{123}
\pstart[\setRL\arabicfont] %
انطلقت الثورة الليبية منذ أكثر من سنة
متوازية مع ثورات الربيع العربي الأخرى التي أطاحت ببعض حكام البلدان
العربية، وما إن وافقت الجامعة العربية وأمريكا وأوروبا على ضرورة تدخل
\edtext{\txarb{منظمة حلف شمال الأطلسي}}{\Afootnote{الأمم المتحدة}}
في المسألة حتى دخلت تركيا على الخط وقامت بإيصال مواد إغاثية وعسكرية
دعما للثوار الليبيين. وهذا التدخل التركي يعكس الواقع الجديد في ملف
العلاقات الخارجية بين تركيا والشرق الأوسط، ألا وهو \edtext{ تعاظم دور تركيا}{\Afootnote{ تزايد الدور التركي}} في العالم الناطق بلغة الضاد، حيث باتت تركيا تنتهج سياسة خارجية نشطة في القضايا الشائكة والعالقة في \edtext{\txarb{الشرق
الأوسط}}{\Afootnote{الشرق عموما}} ومع هذا الاتجاه الجديد في
سياسات تركيا، ينبغي تحديد معالم دور تركيا الجديد، ومحددات هذا الدور
في ضوء التغيرات الأخيرة في المنطقة، وكذلك تحديد آراء العرب حيال
الموقف التركي في الشرق الأوسط.
\pend
\endnumbering
\end{document}

