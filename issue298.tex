\documentclass{scrbook}

\usepackage{fontspec}
\setmainfont{Linux Libertine O}
\usepackage{xunicode}
\usepackage{polyglossia}
\setmainlanguage{german}

\usepackage[xindy,splitindex]{indextools}%On précise qu'on veut utiliser xindy + splitindex
\indexsetup{level=\section*,toclevel=section,noclearpage,othercode=\footnotesize}
\usepackage[series={A},noeledsec,nofamiliar,noend,xindy,xindy+hyperref]{eledmac}
\usepackage[hyperindex=false]{hyperref}

\makeindex[name=verb,intoc=true,title=Index verborum]

\begin{document}

\beginnumbering
\pstart
In principio\edindex[verb]{principium} erat verbum et verbum\edindex[verb]{verbum} erat apud Deum et Deus erat verbum, hoc erat in principio apud Deum.
 Omnia per ipsum facta sunt, et sine ipso factum est nihil\edindex[verb]{nihil}.
 Quod factum est in ipso vita erat, et vita erat lux hominum.
 Et lux\edindex[verb]{lux} in tenebris lucet, et tenebrae eam non comprehenderunt.

 \pend
 \pstart
 Fuit homo\edindex[verb]{homo} missus a Deo\edindex[verb]{deus} cui nomen Johannes.
 Hic venit in testimonium ut testimonium perhiberet de lumine ut omnes crederent per illum.
 Non erat ille lux, sed ut testimonium perhiberet de lumine.
 Erat lux vera quae illuminat omnem hominem venientem in hunc mundum\edindex[verb]{mundus}, in mundo erat et mundus per ipsum factus est et mundus eum non cognovit.

 \pend
 \endnumbering

 \printindex[verb]

 \end{document}
