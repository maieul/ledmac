\documentclass[a4paper,twoside,openright,11pt,final]{memoir}
\usepackage{polyglossia}\usepackage{fontspec}
\usepackage[series={A,B},noledgroup]{reledmac}
\usepackage
    [sameparallelpagenumber]
    {reledpar}
\setgoalfraction{.8}
\linenumincrement*{5}
\firstlinenum*{0}
\setRlineflag{d}    
\setdefaultlanguage{latin}
\usepackage{lipsum,kantlipsum}
\begin{document}

    \begin{pages}
        \begin{Leftside} 
            \beginnumbering
                \pstart[\chapter{Pars sinistra}]
Lorem ipsum dolor sit amet, consectetuer adipiscing elit. Ut purus elit,
vestibulum ut, placerat ac, adipiscing vitae, felis. Curabitur dictum gravida
mauris. Nam arcu libero, nonummy eget, consectetuer id, vulputate a, magna.
Donec vehicula augue eu neque. Pellentesque habitant morbi tristique
senectus et netus et malesuada fames ac turpis egestas. Mauris ut leo. Cras
viverra metus rhoncus sem. Nulla et lectus vestibulum urna fringilla ultrices.
Phasellus eu tellus sit amet tortor gravida placerat. Integer sapien est, iaculis
in, pretium quis, viverra ac, nunc. Praesent eget sem vel leo ultrices bibendum.
Aenean faucibus. Morbi dolor nulla, malesuada eu, pulvinar at, mollis
 ac, nulla. Curabitur auctor semper nulla. Donec varius orci eget risus. Duis
nibh mi, congue eu, accumsan eleifend, sagittis quis, diam. Duis eget orci sit
amet orci dignissim \edtext{rutrum}{\Aendnote{Vedi a pag. 200.}}.

Nam dui ligula, fringilla a, euismod sodales, sollicitudin vel, wisi. Morbi
auctor lorem non justo. Nam lacus libero, pretium at, lobortis vitae, ultricies
 et, tellus. Donec aliquet, tortor sed accumsan bibendum, erat ligula aliquet
magna, vitae ornare odio metus a mi. Morbi ac orci et nisl hendrerit mollis.
Suspendisse ut massa. Cras nec ante. Pellentesque a nulla. Cum sociis
natoque penatibus et magnis dis parturient montes, nascetur ridiculus mus.
Aliquam tincidunt urna. Nulla ullamcorper vestibulum turpis. Pellentesque
 cursus luctus \edtext{mauris}{\Aendnote{Voce a pag. 488.}}.

Nulla malesuada porttitor diam. Donec felis erat, congue non, volutpat
at, tincidunt tristique, libero. Vivamus viverra fermentum felis. Donec nonummy
pellentesque ante. Phasellus adipiscing semper elit. Proin fermentum
massa ac quam. Sed diam turpis, molestie vitae, placerat a, molestie nec, leo.
 Maecenas lacinia. Nam ipsum ligula, eleifend at, accumsan nec, suscipit a,
ipsum. Morbi blandit ligula feugiat magna. Nunc eleifend consequat lorem.
Sed lacinia nulla vitae enim. Pellentesque tincidunt purus vel magna. Integer
non enim. Praesent euismod nunc eu purus. Donec bibendum quam in tellus.
Nullam cursus pulvinar lectus. Donec et mi. Nam vulputate metus eu enim.
 Vestibulum pellentesque felis eu \edtext{massa}{\Aendnote{massa}}.

                \pend
            \endnumbering
        \end{Leftside}

        \begin{Rightside} 
            \beginnumbering
                \pstart[\chapter{Right side}]
As any dedicated reader can clearly see, the Ideal of practical reason is a
representation of, as far as I know, the things in themselves; as I have shown
elsewhere, the phenomena should only be used as a canon for our understanding.
The paralogisms of practical reason are what first give rise to the
architectonic of practical reason. As will easily be shown in the next section, 
reason would thereby be made to contradict, in view of these considerations,
the Ideal of practical reason, yet the manifold depends on the phenomena.
Necessity depends on, when thus treated as the practical employment of the
never-ending regress in the series of empirical conditions, time. Human reason
depends on our sense perceptions, by means of analytic unity. There can be 
no doubt that the objects in space and time are what first give rise to human
\edtext{reason}{\Aendnote{ragione}}.

Let us suppose that the noumena have nothing to do with necessity, since
owledge of the Categories is a posteriori. Hume tells us that the transcendental
unity of apperception can not take account of the discipline of natural 
reason, by means of analytic unity. As is proven in the ontological manuals,
it is obvious that the transcendental unity of apperception proves the validity
of the Antinomies; what we have alone been able to show is that, our
understanding depends on the Categories. It remains a mystery why the Ideal
stands in need of reason. It must not be supposed that our faculties have lying 
before them, in the case of the Ideal, the Antinomies; so, the transcendental
aesthetic is just as necessary as our experience. By means of the Ideal, our
sense perceptions are by their very nature \edtext{contradictory}{\Aendnote{contraddittorio}}.

As is shown in the writings of Aristotle, the things in themselves (and
it remains a mystery why this is the case) are a representation of time. Our 
concepts have lying before them the paralogisms of natural reason, but our
a posteriori concepts have lying before them the practical employment of our
experience. Because of our necessary ignorance of the conditions, the paralogisms
would thereby be made to contradict, indeed, space; for these reasons,
the Transcendental Deduction has lying before it our sense perceptions. (Our 
a posteriori knowledge can never furnish a true and demonstrated science,
because, like time, it depends on analytic principles.) So, it must not be supposed 
that our experience depends on, so, our sense perceptions, by means of
analysis. Space constitutes the whole content for our sense perceptions, and
time occupies part of the sphere of the Ideal concerning the existence of the 
objects in space and time in \edtext{general}{\Aendnote{generale}}.
                \pend
            \endnumbering
        \end{Rightside}
    \end{pages}
\Pages


            \beginnumbering
                \pstart[\chapter{text side}]
As any dedicated reader can clearly see, the Ideal of practical reason is a
representation of, as far as I know, the things in themselves; as I have shown
elsewhere, the phenomena should only be used as a canon for our understanding.
The paralogisms of practical reason are what first give rise to the
architectonic of practical reason. As will easily be shown in the next section, 
reason would thereby be made to contradict, in view of these considerations,
the Ideal of practical reason, yet the manifold depends on the phenomena.
Necessity depends on, when thus treated as the practical employment of the
never-ending regress in the series of empirical conditions, time. Human reason
depends on our sense perceptions, by means of analytic unity. There can be 
no doubt that the objects in space and time are what first give rise to human
\edtext{reason}{\Aendnote{text ragione}}.

Let us suppose that the noumena have nothing to do with necessity, since
owledge of the Categories is a posteriori. Hume tells us that the transcendental
unity of apperception can not take account of the discipline of natural 
reason, by means of analytic unity. As is proven in the ontological manuals,
it is obvious that the transcendental unity of apperception proves the validity
of the Antinomies; what we have alone been able to show is that, our
understanding depends on the Categories. It remains a mystery why the Ideal
stands in need of reason. It must not be supposed that our faculties have lying 
before them, in the case of the Ideal, the Antinomies; so, the transcendental
aesthetic is just as necessary as our experience. By means of the Ideal, our
sense perceptions are by their very nature \edtext{contradictory}{\Aendnote{contraddittorio}}.

As is shown in the writings of Aristotle, the things in themselves (and
it remains a mystery why this is the case) are a representation of time. Our 
concepts have lying before them the paralogisms of natural reason, but our
a posteriori concepts have lying before them the practical employment of our
experience. Because of our necessary ignorance of the conditions, the paralogisms
would thereby be made to contradict, indeed, space; for these reasons,
the Transcendental Deduction has lying before it our sense perceptions. (Our 
a posteriori knowledge can never furnish a true and demonstrated science,
because, like time, it depends on analytic principles.) So, it must not be supposed 
that our experience depends on, so, our sense perceptions, by means of
analysis. Space constitutes the whole content for our sense perceptions, and
time occupies part of the sphere of the Ideal concerning the existence of the 
objects in space and time in \edtext{general}{\Aendnote{generale}}.
                \pend
            \endnumbering

\clearpage

\doendnotes{A}

\end{document}
