\documentclass{scrbook}
\usepackage{eledmac}

\firstlinenum{0} \linenumincrement{1}

\newcounter{inbetween}

\makeatletter
\def\printlines#1|#2|#3|#4|#5|#6|#7|{%
\setcounter{inbetween}{#2}
\stepcounter{inbetween}
\begingroup
\setprintlines{#1}{#2}{#3}{#4}{#5}{#6}%
\ifnum#5>\value{inbetween}\hspace{-1em}{#2}ff. \else       % This line is new
\ifnum#5=\value{inbetween}\hspace{-0.5em}{#2}f. \else
\ifnum#1=-1 \hspace{-0.5em}{#2}/\arabic{inbetween} \else \ifnum#5=#2 \hspace{-0.5em}{#2} \fi \fi \fi
\endgroup}
\makeatother

\newcommand{\insertpartmark}{\edtext{}{\linenum{-1}\Afootnote[nosep]{\hspace{-1em}New chapter mark introduced in 3rd edition}}}

\begin{document}
\beginnumbering      
\pstart
\edtext{What}{\Afootnote{Whit}} a \edtext{piece}{\Afootnote{peace}} of work is man! how noble in reason! 
\pend
\pstart
Dann bekleidete sie mit seinen weichen, weiten Filz
pantoffeln, in denen sie versanken, \edtext{ihre Fuesschen,}{\Afootnote{test}}
welche von Kuessen verwundet waren, und stampfte gravitaetisch
mit einer sehr wuerdigen und kaiserlichen Miene und sang, indem sie mit dem Kopfe den Takt dazu pendelte, ein altes feierliches Kirchenlied. Aber plötzlich, aufrecht auf einem Beine, schnellte sie mit dem Schwunge des andern den Schuh hoch, um ihn durch eine flinke und zuversichtliche Gebaerde wieder aufzufangen. In dieser anmutigen Pose verweilte sie. Sie stiess den Laden nach dem Garten auf, aus welchem der Flieder suesse Gruesse schickte.{\insertpartmark}
\pend

\endnumbering
\end{document}