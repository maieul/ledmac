\documentclass{article}
%For this case I think it's best if it ignores all the \Xend settings and just writes it all out.

\usepackage[utf8]{inputenc}
\usepackage[series={A}]{eledmac}
\usepackage{blindtext}
\renewcommand*{\printnpnum}[1]{%
p.~#1, l.}
\Xendtwolines{sq.}
%\Xendmorethantwolines{sqq.}
\Xendtwolinesbutnotmore
\begin{document}
\beginnumbering
\pstart \edtext{Lorem}{\Aendnote{1ligne}\Afootnote{1ligne}} ipsum dolor amet, \edtext{sit transit gloria mundi, veni vidi vici, le petit chat est mort, trouver des exemples dans ma présence c'est compliqué.
 \edtext{Bon on continue.}{\Afootnote{Sur 2 lignes}\Aendnote{Sur 2 lignes}}
 Allez, je vais y aller ! Et ca recommence ! Franchement c'est fatiguant}{\Afootnote{Sur trois lignes}\Aendnote{Sur trois lignes}}
\pend

\pstart \blindtext \edtext{Lorem}{\linenum{|||2|8}%
\Aendnote{Test: Next page, smaller line number!}} \blindtext[2]
\pend

\pstart \blindtext \edtext{Lorem}{\linenum{|||3|45}%
\Aendnote{Test: Next page, same line number!}} \blindtext[2]
\pend

\pstart \blindtext \edtext{Lorem}{\linenum{|||3|77}%
\Aendnote{Test: Next page, line number plus one!}} \blindtext[2]
\pend

\pstart \blindtext \edtext{Lorem}{\linenum{|||4|110}%
\Aendnote{Test: Next page, higher line number!}} \blindtext[2]
\pend

\endnumbering

\doendnotes{A}
\end{document}
