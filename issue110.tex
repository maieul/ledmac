\documentclass{article}
\usepackage{eledmac}

\begin{document}
\makeatletter

\apptocmd{\no@expands}{\let\sameword\sameword@inedtext}{}{}


\def\sameword@inedtext#1{%
  \ifx\sw@@list\empty%
    \def\the@sw{999}%
  \else%
    \gl@p\sw@@list\to\the@sw%
  \fi%
  \ifcsdef{sw@#1@\the\absline@num}{%
  \ifnumgreater{\csuse{sw@#1@\the\absline@num}}{1}%
    {\showwordrank{#1}{\the@sw}}%
    {#1}%
  }{#1}%
}

\newif\if@edtext
\renewcommand{\edtext}[2]{\leavevmode
  \@edtexttrue
  \begingroup
    \global\renewcommand{\@tag}{\no@expands #1}%%
    \set@line
    \global\insert@count=0
    \ignorespaces #2\relax
    \@ifundefined{xpg@main@language}{%if not polyglossia
       \flag@start}%
       {\if@RTL\flag@end\else\flag@start\fi% With polyglossia, you must track whether the language reads left to right (English) or right to left (Arabic).
       }%
  \endgroup
  \showlemma{#1}%
  \ifx\end@lemmas\empty \else
    \gl@p\end@lemmas\to\x@lemma
    \x@lemma
    \global\let\x@lemma=\relax
  \fi
  \@ifundefined{xpg@main@language}{%if not polyglossia
       \flag@end}%
       {\if@RTL\flag@start\else\flag@end\fi% With polyglossia, you must track whether the language reads left to right (English) or right to left (Arabic).
       }%
  \global\@noneed@Footnotefalse%
  \@edtextfalse
  }


\newcommand{\sameword}[1]{%
	\csnumgdef{sw@#1}{\csuse{sw@#1}+1}%
	\protected@write\linenum@out{}{\string\@sw{#1}{\csuse{sw@#1}}}%
	\ifx\sw@list\empty%
	  \xdef\sw@num{0}%
	\else%
	  \gl@p\sw@list\to\@tempb%
      \if@edtext%
        \xright@appenditem{\@tempb}\to\sw@@list%
      \fi%
      \global\let\@tempb=\undefined%
	  %
	\fi%
  #1%
}
\newcommand{\@sw}[2]{%
  \csxdef{sw@#1@\the\absline@num}{#2}
  \numdef{\prev@line}{\the\absline@num-1}
  \ifcsundef{sw@#1@\prev@line}{
     \csnumgdef{sw@#1@\prev@line}{\csuse{sw@#1@\the\absline@num}-1}
    }{}
    \numdef{\the@@@sw}{#2-\csuse{sw@#1@\prev@line}}%%%%%%%
  \xright@appenditem{\the@@@sw}\to\sw@list
}
\def\@sw@inref#1#2{}

\list@create{\sw@@list}
\list@create{\sw@list}

\renewcommand*{\@ref@reg}[2]{%
  \global\insert@count=#1\relax
  \loop\ifnum\insert@count>\z@
    \xright@appenditem{\the\absline@num}\to\insertlines@list
    \global\advance\insert@count \m@ne
  \repeat
  \begingroup
    \let\@ref=\dummy@ref
    \let\@lopL\@gobble
    \let\page@action=\relax
    \let\sub@action=\relax
    \let\set@line@action=\relax
    \let\@lab=\relax
    \let\@sw\@sw@inref
    #2
    \global\endpage@num=\page@num
    \global\endline@num=\line@num
    \global\endsubline@num=\subline@num
  \endgroup
    \xright@appenditem%
      {\the\page@num|\the\line@num|%
       \ifsublines@ \the\subline@num \else 0\fi|%
       \the\endpage@num|\the\endline@num|%
       \ifsublines@ \the\endsubline@num \else 0\fi}\to\line@list
  #2}





\newcommand{\showwordrank}[2]{#1\ifnumgreater{#2}{0}{\textsuperscript{#2}}{}}

\begin{ledgroup}

\beginnumbering

\pstart
Tertia est – supposito, quod sint passiones totius coniuncti, cum non insint toti nisi ratione partium –, cui parti corporis uel animae debeant attribui, \sameword{aut} utrum uegetatiuae \sameword{aut} \sameword{aut} sensitiuae \edtext{\sameword{aut}}{\Afootnote{sncf}}
intellectiuae.


\pend

\endnumbering
\end{ledgroup}


\makeatother

\end{document}
