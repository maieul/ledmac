\documentclass[11pt,a4paper]{article}
\usepackage[noeledsec,noend,series={A,C}]{reledmac}
\usepackage{polyglossia}
\parindentX
\Xparindent
\mpfnpos{familiar-critical}
\mpfnpos{{A}{familiar},{A}{critical},{C}{familiar}}
\setdefaultlanguage{spanish}
\setotherlanguage[variant=ancient]{greek}
\setmainfont{IFAO-Grec Unicode}
\newcommand{\dott}{\char"E5CE}
%
\newlength\parindentbackup
\parindentbackup=\parindent
\begin{document}%
\begin{ledgroup}
    \begin{center}
\parindent\parindentbackup
        [5]\footnoteA{\textit{P. Oxy. 7} [Grenf.-Hunt] [I]; \textit{P. G. C. inv. 105} [B. F. O.] [II]; \textit{P. Oxy. 2289}, fr. 6 [Lobel] [III]; Obb. 2016 [I+II+III].}
   \end{center}
    \beginnumbering
    \pstart
    ⊗ \edtext{Here}{\Afootnote{ποτνιαι\dott\dott[ \textit{Π3}}} \edtext{πότνιαι}{\Afootnote{Κύπρι καὶ \textit{suppl.} Earle \textit{sec.} Smyth; Χρύσιαι Edm.; ὦ φίλαι Blass [I]}}\footnoteC{Aunque el \textit{P. G. C. inv. 105} fue editado y publicado por Obbink (2014a), ya Diels (1898) había propuesto la lección πότνιαι en su \textit{recensio} a la 1ª parte de la colección \textit{The Oxyrhyncus Papyri}. El papiro confirmaría su propuesta años más tarde.}
    \pend
\endnumbering
\end{ledgroup}
    %
\end{document}
