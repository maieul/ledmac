\documentclass[11pt,a4paper]{article}
\usepackage[noeledsec,noend,series={A,C}]{reledmac}
\parindentX
\Xparindent
\mpfnpos{familiar-critical}
\mpfnpos{{A}{familiar},{A}{critical},{C}{familiar}}
\newlength\parindentbackup
\parindentbackup=\parindent
\def\dott{...}
\begin{document}%
\begin{ledgroup}
        [5]\footnoteA{\textit{P. Oxy. 7} [Grenf.-Hunt] [I]; \textit{P. G. C. inv. 105} [B. F. O.] [II]; \textit{P. Oxy. 2289}, fr. 6 [Lobel] [III]; Obb. 2016 [I+II+III].}
    \beginnumbering
    \pstart
    \edtext{Here}{\Afootnote{potniai\dott\dott[ \textit{Pi3}}} \edtext{pontniai}{\Afootnote{Cypri kai \textit{suppl.} Earle \textit{sec.} Smyth; Chrusiai Edm.; o filai Blass [I]}}\footnoteC{Aunque el \textit{P. G. C. inv. 105} fue editado y publicado por Obbink (2014a), ya Diels (1898) había propuesto la lección potniai en su \textit{recensio} a la 1ª parte de la colección \textit{The Oxyrhyncus Papyri}. El papiro confirmaría su propuesta años más tarde.}

    \pend
\endnumbering
\end{ledgroup}
    %
\end{document}
