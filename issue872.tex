\documentclass[11pt,twoside]{book}

\usepackage{fontspec}
\usepackage{libertinus}
\usepackage[parapparatus]{reledmac}


\Xarrangement[A]{paragraph}
\Xarrangement[B]{paragraph}
\Xarrangement[C]{paragraph}
\Xarrangement[D]{paragraph}

\newcommand{\choiceline}[2]{\linenum{|#1|#2||#1|#2}}
\sublinenumberstyle{alph}
\let\fullstop\relax

\newcommand{\sktt}{\setlength{\leftskip}{15mm}\setlength{\rightskip}{15mm}\itshape}
\newcommand{\skte}{\setlength{\leftskip}{15mm}\setlength{\rightskip}{15mm}\itshape\skipnumbering}


\setlength{\parindent}{0pt}
\setstanzaindents{2,2,2,2}
\setcounter{stanzaindentsrepetition}{0}
\numberstanzafalse


\Xstanza
\Xstanzaseparator{.}

\hidenumberingonrightpage
\lineation{pstart}
\firstlinenum{1}
\linenumincrement{1}
\firstsublinenum{1}
\sublinenumincrement{1}


\begin{document}

\beginnumbering
  \autopar

\skipnumbering
\centering{||} O{||} \textit{avighnam astu}{||} O{||}

\pstart
\noindent{}\edtext{saṅ}{\Bfootnote{em.; sa LOr}} saṅgrahakāri sira movus{|} liṅnira{|}
\pend

\stanza
\hidenumbering\itshape\edtext{śūnyaś}{\choiceline{1}{1}\Afootnote{em.; śūnya A}}  ca \edtext{nirbbāṇādhikaḥ{|} śivāṅgatve}{\linenum{|1|1||1|2|}\Afootnote{em; nirbbhāṇādhika{|} śśivaṅgatve A, B}} nirīkṣyate{|}&
\hidenumbering\itshape kutaḥ tadvākyam atulaṁ{|} śrutvā devo \edtext{’vatiṣṭhati}{\choiceline{1}{4}\Afootnote{em; vatiṣṭha{|} ca A, B}} {||} \normalshape 1\&[ ]


\pstart
nāhan takvanaknaniṅ hulun ri bhaṭāra{|} hana ya pada śūnya{|} ya sinaṅguh ka\-mo\-kṣan{|} ṅa{|} viśeṣa ya{|} ya śiva ṅaranya{|} \textit{nirīkṣyate}{|} katon pva ya de saṅ \edtext{yogīśvara}{\lemma{\ldots{}gīśvara}\Cfootnote{B om.}}{|} sājñā bhaṭāra{|} an maṅkana kottamaniṅ vuvus saṅ ṛṣi{|} ya ta kaṛṅĕ de bhaṭāra{||} 1
\pend

\pstart
\edtext{\textit{\edtext{kenopāyena bhagavan}{\choiceline{2}{1}\Afootnote{em.; meyopāyena bhagavān LOr}\Dfootnote{unmetrical (\textit{bha-vipulā} with short second syllable)}}}{|}\textit{sukhayogasya lakṣaṇaṁ}}{\linenum{|2|1||2|2|}\lemma{kenopāyena \ldots\ lakṣaṇam}\Cfootnote{cf. BhK 25.1}}{|} \edtext{nihan}{\Bfootnote{conj.; nāhan A}} takvananiṅ hulun ri kita bha\-ṭā\-ra{|} ndya kunaṅ \edtext{upāyanikā}{\Bfootnote{corr.; opāyanikā A}} saṅ paṇḍita{|} mataṅyan kapaṅguhāvaknikaṅ sukhādhyātmika{||} 2
\pend

\textit{deva uvāca}{|} \edtext{mojar ta}{\Bfootnote{mojar PDok}} bhaṭāra{|} liṅnira{||}

\stanza
\hidenumbering\itshape
svaśarīre mahāyogī{|} \edtext{paśyate}{\choiceline{3}{2}\Dfootnote{for dr̥śyate?}} hr̥dayāntare{|}&
\hidenumbering\itshape\edtext{vākyan}{\choiceline{3}{3}\Dfootnote{vācyan ?}}  te parameśānam{|} \edtext{sūryyāyutasamaprabham}{\choiceline{3}{4}\Afootnote{em.; °tamamapra° all mss.}}{||} \normalshape 3\&[ ]

\pstart
nihan \edtext{vuvusaniṅ}{\Bfootnote{vuvusniṅ PDok}} hulun i kita{|} ikā saṅ mahāyogī{|} sira tumon bhaṭāra \edtext{parameśvara}{\Bfootnote{prameśvara PDok}}{|} sateja sira lāvan tejaniṅ āditya sāyuta{|} \edtext{ṅkā ri śarīranira}{\Bfootnote{ṅkā riṅ śarīraniraṅ PDok}}  mvaṅ ri hatinira{||} 3
\pend

\endnumbering


\end{document}

