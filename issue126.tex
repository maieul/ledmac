\documentclass[12pt,a4paper]{memoir}
\usepackage[french]{babel}
\usepackage[no-math]{fontspec}
\usepackage{amsmath}

\defaultfontfeatures{Mapping=tex-text,RawFeature={+hlig;+onum}}
\setmainfont{Linux Libertine O}
\setsansfont{Linux Biolinum O}
\setmonofont{Linux Libertine Mono O}
\newfontfamily\uncial{P39}


%%%begin
\usepackage[final]{eledmac}
\usepackage{eledpar}
\lineation{page}
\footparagraph{A}
\txtbeforeXnotes[A]{\textsc{Test.}:\space}
\symlinenum[B]{$\parallel$}
\numberonlyfirstinline[B]
\numberonlyfirstintwolines[B]
\nolemmaseparator[B]
\inplaceoflemmaseparator[B]{.5em}
\nonbreakableafternumber
\footparagraph{B}

\Xnotenumfont{\bfseries}
\renewcommand*{\Rlineflag}{}
\renewcommand*{\bodyfootmarkA}{%
\hbox{\textsuperscript{\thefootnoteA}}}
%\renewcommand*{\goalfraction}{0.8}% mais .825 passe aussi
%%%end


%%%
\usepackage{polyglossia}
\setdefaultlanguage{french}
\setotherlanguage{arabic}

%\newfontfamily\arabicfont[Script=Arabic]{Amiri}
\usepackage[voc,fdf2noalif]{arabxetex}
\newcommand{\ta}{\textarab}
\newcommand{\tanv}[1]{\textarab[novoc]{#1}}
\catcode`_=11
\catcode`^=11

\let\aemph\veryundefinedcommand
%%%

\begin{document}
\begin{pages}
\begin{Leftside}
\ledsectnotoc

\footnote{S}
\beginnumbering

\firstlinenum{100}

\pstart\vspace{2.1\baselineskip}
\markboth{5}{}\textbf{5}
Hippocrate dit: Si les saisons de l'année sont conformes à leur ordre
et si dans chacune d'elles se trouve ce qui doit s'y trouver, les
maladies qui y apparaissent sont régulières et de crise
favorable\footnoteA{Je fais une note ici. Comme on le voit,
ça marche parfaitement.}; mais si les saisons de l'année ne sont pas
conformes à leur ordre, les maladies qui y apparaissent sont irrégulières
et de crise défavorable. C'est ce qui <se produisit> dans la cité de Périnthe
lorsque quelque chose était en défaut ou en excès soit dans les vents,
ou dans l'absence de vent, soit dans la pluie ou l'absence de pluie,
soit dans la chaleur, soit dans le froid, puisque le printemps était
en général très sain et de très faible mortalité.
\pend

\pstart\vspace{2.1\baselineskip}
\markboth{5}{}\textbf{5}
Hippocrate dit: Si les saisons de l'année sont conformes à leur ordre
et si dans chacune d'elles se trouve ce qui doit s'y trouver, les
maladies qui y apparaissent sont régulières et de crise favorable;
mais si les saisons de l'année ne sont pas conformes à leur ordre, les
maladies qui y apparaissent sont irrégulières et de crise
défavorable. C'est ce qui <se produisit> dans la cité de Périnthe
lorsque quelque chose était en défaut ou en excès soit dans les vents,
ou dans l'absence de vent, soit dans la pluie ou l'absence de pluie,
soit dans la chaleur, soit dans le froid, puisque le printemps était
en général très sain et de très faible mortalité.
\pend

\pstart\vspace{2.1\baselineskip}
\markboth{5}{}\textbf{5}
Hippocrate dit: Si les saisons de l'année sont conformes à leur ordre
et si dans chacune d'elles se trouve ce qui doit s'y trouver, les
maladies qui y apparaissent sont régulières et de crise favorable;
mais si les saisons de l'année ne sont pas conformes à leur ordre, les
maladies qui y apparaissent sont irrégulières et de crise
défavorable. C'est ce qui <se produisit> dans la cité de Périnthe
lorsque quelque chose était en défaut ou en excès soit dans les vents,
ou dans l'absence de vent, soit dans la pluie ou l'absence de pluie,
soit dans la chaleur, soit dans le froid, puisque le printemps était
en général très sain et de très faible mortalité.
\pend

\pstart\vspace{2.1\baselineskip}
\markboth{5}{}\textbf{5}
Hippocrate dit: Si les saisons de l'année sont conformes à leur ordre
et si dans chacune d'elles se trouve ce qui doit s'y trouver, les
maladies qui y apparaissent sont régulières et de crise favorable;
mais si les saisons de l'année ne sont pas conformes à leur ordre, les
maladies qui y apparaissent sont irrégulières et de crise
défavorable. C'est ce qui <se produisit> dans la cité de Périnthe
lorsque quelque chose était en défaut ou en excès soit dans les vents,
ou dans l'absence de vent, soit dans la pluie ou l'absence de pluie,
soit dans la chaleur, soit dans le froid, puisque le printemps était
en général très sain et de très faible mortalité.
\pend

\pstart\vspace{2.1\baselineskip}
\markboth{5}{}\textbf{5}
Hippocrate dit: Si les saisons de l'année sont conformes à leur ordre
et si dans chacune d'elles se trouve ce qui doit s'y trouver, les
maladies qui y apparaissent sont régulières et de crise favorable;
mais si les saisons de l'année ne sont pas conformes à leur ordre, les
maladies qui y apparaissent sont irrégulières et de crise\footnoteA{Et voici
une nouvelle note: ça ne marche plus sur une nouvelle paire de pages}.
défavorable. C'est ce qui <se produisit> dans la cité de Périnthe
lorsque quelque chose était en défaut ou en excès soit dans les vents,
ou dans l'absence de vent, soit dans la pluie ou l'absence de pluie,
soit dans la chaleur, soit dans le froid, puisque le printemps était
en général très sain et de très faible mortalité.
\pend

\pstart\vspace{2.1\baselineskip}
\markboth{5}{}\textbf{5}
Hippocrate dit: Si les saisons de l'année sont conformes à leur ordre
et si dans chacune d'elles se trouve ce qui doit s'y trouver, les
maladies qui y apparaissent sont régulières et de crise favorable;
mais si les saisons de l'année ne sont pas conformes à leur ordre, les
maladies qui y apparaissent sont irrégulières et de crise
défavorable. C'est ce qui <se produisit> dans la cité de Périnthe
lorsque quelque chose était en défaut ou en excès soit dans les vents,
ou dans l'absence de vent, soit dans la pluie ou l'absence de pluie,
soit dans la chaleur, soit dans le froid, puisque le printemps était
en général très sain et de très faible mortalité.
\pend



\endnumbering

\end{Leftside}

\begin{Rightside}
\linenummargin{inner}
\lineation{page}

\beginnumbering


\pstart\markright{\ta{5}}
  \begin{arab}
    (5) qAla 'abuqrA.tu: 'i_d kAnat 'awqAtu 'l-ssanaTi lAzimaTaN
    li-ni.zAmihA wa-kAna fI kulli waqtiN minhA mA yanba.gI 'an yakUna
    fIhi kAna mA ya.hda_tu fIhA mina 'l-'amrA.di .husna 'l-nni.zAmi
    .husna 'l-bu.hrAni. wa-'i_dA kAnat 'awqAtu 'l-ssanaTi .gayra
    lAzimaTiN li-ni.zAmihA \edtext{kAna}{\Bfootnote{E1: om.M}} mA
    ya.hda_tu fIhA mina 'l-'amrA.di .gayra munta.zamiN samja
    'l-bu.hrAni. min _d_alika mA yakUnu bi-madInaTi
    \edtext{bArin_tusa}{\Bfootnote{Vagelpohl/Hallum: \tanv{.bAr.bs} E1
        \tanv{bArIns} M}} matt_A naqa.sa ^say'uN 'aw
    \edtext{zAda}{\Bfootnote{E1: \tanv{AzdAd} M}} fIhA 'immA mina
    'l-rriyA.hi wa-'immA min `adamihA wa-'immA mina 'l-ma.tari
    wa-'immA min `adamihi wa-'immA mina 'l-.harri wa-'immA mina
    'l-bardi \edtext{wa-_tumma}{\Bfootnote{M: \tanv{_tumma} E1}}
    yakUnu 'l-rrabI`u fI 'ak_tari 'l-'amri `al_A 'af.dali .hAlAti
    'l-.s.sa.haTi
    \edtext{wa-'aqalla}{\lemma{\tanv{wa-'aqalla}}\Bfootnote{E1:
        \tanv{wql} M}} mA ya.hda_tu 'l-mawtu.
  \end{arab}
\pend

\pstart\markright{\ta{5}}
  \begin{arab}
    (5) qAla 'abuqrA.tu\footnote{Une note qui devrait être RTL}: 'i_d kAnat 'awqAtu 'l-ssanaTi lAzimaTaN
    li-ni.zAmihA wa-kAna fI kulli waqtiN minhA mA yanba.gI 'an yakUna
    fIhi kAna mA ya.hda_tu fIhA mina 'l-'amrA.di .husna 'l-nni.zAmi
    .husna 'l-bu.hrAni. wa-'i_dA kAnat\footnoteA{Une autre note qui devrait auss être RTL} 'awqAtu 'l-ssanaTi .gayra
    lAzimaTiN li-ni.zAmihA \edtext{kAna}{\Bfootnote{E1: om.M}} mA
    ya.hda_tu fIhA mina 'l-'amrA.di .gayra munta.zamiN samja
    'l-bu.hrAni. min _d_alika mA yakUnu bi-madInaTi
    \edtext{bArin_tusa}{\Bfootnote{Vagelpohl/Hallum: \tanv{.bAr.bs} E1
        \tanv{bArIns} M}} matt_A naqa.sa ^say'uN 'aw
    \edtext{zAda}{\Bfootnote{E1: \tanv{AzdAd} M}} fIhA 'immA mina
    'l-rriyA.hi wa-'immA min `adamihA wa-'immA mina 'l-ma.tari
    wa-'immA min `adamihi wa-'immA mina 'l-.harri wa-'immA mina
    'l-bardi \edtext{wa-_tumma}{\Bfootnote{M: \tanv{_tumma} E1}}
    yakUnu 'l-rrabI`u fI 'ak_tari 'l-'amri `al_A 'af.dali .hAlAti
    'l-.s.sa.haTi
    \edtext{wa-'aqalla}{\lemma{\tanv{wa-'aqalla}}\Bfootnote{E1:
        \tanv{wql} M}} mA ya.hda_tu 'l-mawtu.
  \end{arab}
\pend

\pstart\markright{\ta{5}}
  \begin{arab}
    (5) qAla 'abuqrA.tu: 'i_d kAnat 'awqAtu 'l-ssanaTi lAzimaTaN
    li-ni.zAmihA wa-kAna fI kulli waqtiN minhA mA yanba.gI 'an yakUna
    fIhi kAna mA ya.hda_tu fIhA mina 'l-'amrA.di .husna 'l-nni.zAmi
    .husna 'l-bu.hrAni. wa-'i_dA kAnat 'awqAtu 'l-ssanaTi .gayra
    lAzimaTiN li-ni.zAmihA \edtext{kAna}{\Bfootnote{E1: om.M}} mA
    ya.hda_tu fIhA mina 'l-'amrA.di .gayra munta.zamiN samja
    'l-bu.hrAni. min _d_alika mA yakUnu bi-madInaTi
    \edtext{bArin_tusa}{\Bfootnote{Vagelpohl/Hallum: \tanv{.bAr.bs} E1
        \tanv{bArIns} M}} matt_A naqa.sa ^say'uN 'aw
    \edtext{zAda}{\Bfootnote{E1: \tanv{AzdAd} M}} fIhA 'immA mina
    'l-rriyA.hi wa-'immA min `adamihA wa-'immA mina 'l-ma.tari
    wa-'immA min `adamihi wa-'immA mina 'l-.harri wa-'immA mina
    'l-bardi \edtext{wa-_tumma}{\Bfootnote{M: \tanv{_tumma} E1}}
    yakUnu 'l-rrabI`u fI 'ak_tari 'l-'amri `al_A 'af.dali .hAlAti
    'l-.s.sa.haTi
    \edtext{wa-'aqalla}{\lemma{\tanv{wa-'aqalla}}\Bfootnote{E1:
        \tanv{wql} M}} mA ya.hda_tu 'l-mawtu.
  \end{arab}
\pend

\pstart\markright{\ta{5}}
  \begin{arab}
    (5) qAla 'abuqrA.tu: 'i_d kAnat 'awqAtu 'l-ssanaTi lAzimaTaN
    li-ni.zAmihA wa-kAna fI kulli waqtiN minhA mA yanba.gI 'an yakUna
    fIhi kAna mA ya.hda_tu fIhA mina 'l-'amrA.di .husna 'l-nni.zAmi
    .husna 'l-bu.hrAni. wa-'i_dA kAnat 'awqAtu 'l-ssanaTi .gayra
    lAzimaTiN li-ni.zAmihA \edtext{kAna}{\Bfootnote{E1: om.M}} mA
    ya.hda_tu fIhA mina 'l-'amrA.di .gayra munta.zamiN samja
    'l-bu.hrAni. min _d_alika mA yakUnu bi-madInaTi
    \edtext{bArin_tusa}{\Bfootnote{Vagelpohl/Hallum: \tanv{.bAr.bs} E1
        \tanv{bArIns} M}} matt_A naqa.sa ^say'uN 'aw
    \edtext{zAda}{\Bfootnote{E1: \tanv{AzdAd} M}} fIhA 'immA mina
    'l-rriyA.hi wa-'immA min `adamihA wa-'immA mina 'l-ma.tari
    wa-'immA min `adamihi wa-'immA mina 'l-.harri wa-'immA mina
    'l-bardi \edtext{wa-_tumma}{\Bfootnote{M: \tanv{_tumma} E1}}
    yakUnu 'l-rrabI`u fI 'ak_tari 'l-'amri `al_A 'af.dali .hAlAti
    'l-.s.sa.haTi
    \edtext{wa-'aqalla}{\lemma{\tanv{wa-'aqalla}}\Bfootnote{E1:
        \tanv{wql} M}} mA ya.hda_tu 'l-mawtu.
  \end{arab}
\pend

\pstart\markright{\ta{5}}
  \begin{arab}
    (5) qAla 'abuqrA.tu: 'i_d kAnat 'awqAtu 'l-ssanaTi lAzimaTaN
    li-ni.zAmihA wa-kAna fI kulli waqtiN minhA mA yanba.gI 'an yakUna
    fIhi kAna mA ya.hda_tu fIhA mina 'l-'amrA.di .husna 'l-nni.zAmi
    .husna 'l-bu.hrAni. wa-'i_dA kAnat 'awqAtu 'l-ssanaTi .gayra
    lAzimaTiN li-ni.zAmihA \edtext{kAna}{\Bfootnote{E1: om.M}} mA
    ya.hda_tu fIhA mina 'l-'amrA.di .gayra munta.zamiN samja
    'l-bu.hrAni. min _d_alika mA yakUnu bi-madInaTi
    \edtext{bArin_tusa}{\Bfootnote{Vagelpohl/Hallum: \tanv{.bAr.bs} E1
        \tanv{bArIns} M}} matt_A naqa.sa ^say'uN 'aw
    \edtext{zAda}{\Bfootnote{E1: \tanv{AzdAd} M}} fIhA 'immA mina
    'l-rriyA.hi wa-'immA min `adamihA wa-'immA mina 'l-ma.tari
    wa-'immA min `adamihi wa-'immA mina 'l-.harri wa-'immA mina
    'l-bardi \edtext{wa-_tumma}{\Bfootnote{M: \tanv{_tumma} E1}}
    yakUnu 'l-rrabI`u fI 'ak_tari 'l-'amri `al_A 'af.dali .hAlAti
    'l-.s.sa.haTi
    \edtext{wa-'aqalla}{\lemma{\tanv{wa-'aqalla}}\Bfootnote{E1:
        \tanv{wql} M}} mA ya.hda_tu 'l-mawtu.
  \end{arab}
\pend

\pstart\markright{\ta{5}}
  \begin{arab}
    (5) qAla 'abuqrA.tu: 'i_d kAnat 'awqAtu 'l-ssanaTi lAzimaTaN
    li-ni.zAmihA wa-kAna fI kulli waqtiN minhA mA yanba.gI 'an yakUna
    fIhi kAna mA ya.hda_tu fIhA mina 'l-'amrA.di .husna 'l-nni.zAmi
    .husna 'l-bu.hrAni. wa-'i_dA kAnat 'awqAtu 'l-ssanaTi .gayra
    lAzimaTiN li-ni.zAmihA \edtext{kAna}{\Bfootnote{E1: om.M}} mA
    ya.hda_tu fIhA mina 'l-'amrA.di .gayra munta.zamiN samja
    'l-bu.hrAni. min _d_alika mA yakUnu bi-madInaTi
    \edtext{bArin_tusa}{\Bfootnote{Vagelpohl/Hallum: \tanv{.bAr.bs} E1
        \tanv{bArIns} M}} matt_A naqa.sa ^say'uN 'aw
    \edtext{zAda}{\Bfootnote{E1: \tanv{AzdAd} M}} fIhA 'immA mina
    'l-rriyA.hi wa-'immA min `adamihA wa-'immA mina 'l-ma.tari
    wa-'immA min `adamihi wa-'immA mina 'l-.harri wa-'immA mina
    'l-bardi \edtext{wa-_tumma}{\Bfootnote{M: \tanv{_tumma} E1}}
    yakUnu 'l-rrabI`u fI\footnoteA{Une note qui devrait être RTL} 'ak_tari 'l-'amri `al_A 'af.dali .hAlAti
    'l-.s.sa.haTi
    \edtext{wa-'aqalla}{\lemma{\tanv{wa-'aqalla}}\Bfootnote{E1:
        \tanv{wql} M}} mA ya.hda_tu 'l-mawtu\footnote{Une note qui devrait être RTL}.
  \end{arab}
\pend



\endnumbering

\end{Rightside}

\Pages

\end{pages}
\end{document}