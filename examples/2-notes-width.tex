\documentclass[12pt]{article}
\usepackage{lipsum}
\usepackage{polyglossia}
\usepackage{libertineotf}
\usepackage[a4paper,textwidth=12cm]{geometry}
\setmainlanguage{latin}
\setotherlanguage{english}
\usepackage[noend,noeledsec,noledgroup,series={A,B,C,D}]{reledmac}
\Xarrangement[A]{paragraph}
\Xarrangement[B]{twocol}
\Xarrangement[C]{threecol}
\arrangementX[A]{paragraph}
\arrangementX[B]{twocol}
\arrangementX[C]{threecol}
\Xwidth{\columnwidth+\marginparsep+\ledrsnotewidth}
\widthX{\columnwidth+\marginparsep+\ledrsnotewidth}
\rightnoteupfalse

\AtBeginDocument{%
 \maxhnotesX{0.8\textheight}
 \Xmaxhnotes{0.8\textheight}
}
\Xcolalign{\justifying}
\colalignX{\justifying}
\begin{document}

\begin{english}
\date{}
\title{Setting footnotes width}
\maketitle
\begin{abstract}
This example use  \verb+maxhnotesX+ and \verb+Xmaxhnotes+ to set the width of familiar and critical footnotes. The width is set to \verb+\columnwidth+\marginparsep+\ledrsnotewidth+, that means footnotes go from the left of the text to the right of the sidenotes.
\end{abstract}
\end{english}
\newpage
\beginnumbering
\pstart
\edtext{Lorem}{
    \Afootnote{A\lipsum*[1]}
    \Bfootnote{B\lipsum*[2]}
    \Cfootnote{C\lipsum*[3]}
    \Dfootnote{D\lipsum*[4]}
    }
Dolor
\ledsidenote{A long side note.}
\lipsum*[1]
\pend
\newpage
\pstart
\lipsum*[2]\ledsidenote{A long side note.}\footnoteA{A\lipsum*[5]}\footnoteB{B\lipsum*[6]}\footnoteC{C\lipsum*[7]}\footnoteD{D\lipsum*[8]}
\pend
\endnumbering
\end{document}
