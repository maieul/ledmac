\documentclass{article}
\usepackage{polyglossia,fontspec}
\usepackage{libertineotf}
\setmainlanguage{latin}
\setotherlanguage{english}

\usepackage[series={A,B},nocritical,noeledsec,nofamiliar,noledgroup]{reledmac}
\Xendparagraph[B]
\begin{document}

\begin{english}
\date{}
\title{Critical endnotes}
\maketitle
\begin{abstract}
This file provides examples of critical endnotes with reledmac. 
A critical note is associated to a lemma, marked by \verb+\edtext+, and referenced by the line and page numbers of the lemma.
If a critical notes refers to a long lemma, we can use \verb+lemma+ to obtain an abbreviated form.

Here we use two series of critical notes. 
\begin{itemize}
\item Each note of series A has its own paragraph. 
\item The notes of series B are arranged in the same paragraph.
\end{itemize}
\end{abstract}
\end{english}

\beginnumbering
\pstart
\edtext{Lorem}{
  \Aendnote{A critical note}
  \Aendnote{Critical note in series A}
  \Aendnote{Critical note in series A}
  \Bendnote{loram}}
\edtext{ipsum}{
  \Aendnote{An other critical note}
  \Bendnote{Other critical note in series A}}
 dolor sit amet, consectetur adipiscing elit. 
 \edtext{Fusce sed dolor libero. Aenean rutrum vestibulum lacus ut pretium. Fusce et auctor lectus. Ut et commodo quam, quis gravida orci. Nullam at risus elementum, suscipit enim a, pellentesque mi}
 {\lemma{Fusce\ldots mi}
 \Aendnote{A long critical note}
 \Bendnote{omit}}. 
Morbi \edtext{commodo}{\Bendnote{quommodo}}, ligula vel consectetur accumsan, massa metus egestas velit, eu fringilla leo ante in turpis. \edtext{Vivamus}{\Bendnote{Vivit}} ut tellus sollicitudin, facilisis ipsum sit amet, tincidunt odio. Maecenas tincidunt dolor sed ante blandit tincidunt. Etiam vulputate ultricies facilisis.
\pend
\endnumbering

\section{A series}
\doendnotes{A}

\section{B series}
\doendnotes{B}


\end{document}