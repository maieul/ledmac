\documentclass{article}
\usepackage[T1]{fontenc}
\usepackage[osf,nomath,p]{libertinus}
\usepackage{microtype}
\usepackage[pdfusetitle,hidelinks]{hyperref}
\usepackage[english, main=latin]{babel}
\babeltags{english = english}

\usepackage[series={A,B},noend,noeledsec,noledgroup]{reledmac}
\sethangingsymbol{[\,}
\firstlinenum{1}
\linenumincrement{1}
\setcounter{stanzaindentsrepetition}{0}
\setstanzaindents{8,0,0}
\setcounter{stanzaindentsrepetition}{1}

\begin{document}
\begin{english}
\date{}

\title{Stanza with double line numbering}
\maketitle

\begin{abstract}
This file provides an example of typesetting verse with reledmac using a double line numbering system.

The new edition adds a starting line of verse not present in the previous one, so we need to make an addition within the margin and textsuperscript the numbering from the older edition. In the centre, we split what was in the first edition one line of verse into two.



To mark the old numbering, we use \verb+\linenumannotation+. For maximum clarity, all line numbers are shown thanks to the settings of \verb+\firstlinenum+ and \verb+\linenumincrement+ in the preamble.

\end{abstract}
\end{english}

\beginnumbering

\stanza
\edlabel{begin:1}\edtext{Lorem}{\lemma{Lorem\ldots nisis}\xxref{begin:1}{end:1}\Afootnote{A note on four lines of verse, the first of which was missing in the first edition}} ipsum dolor sit amet, consectetur adipisicing elit,&
\linenumannotation{1}sed do eiusmod tempor incididunt ut labore et dolore&
\linenumannotation{2}magna aliqua. Ut enim ad minim veniam, quis nostrud&
\linenumannotation{3}exercitation ullamco laboris nisi\edlabel{end:1}&
\linenumannotation{4}\edtext{ut aliquip}{\Afootnote{ut aliliquip}} consequat ut aliquip consequat irure dolor in reprehenderit irure dolor in reprehenderit&
\linenumannotation{5}\edtext{Duis aute}{\Bfootnote{Some comments}} irure dolor in reprehenderit&
\linenumannotation{6}\edlabel{begin:2}\edtext{in}{\xxref{begin:2}{end:2}\lemma{in\ldots deserunt}\Afootnote{Theses two lines of verse were one single line in the first edition}} voluptate velit esse cillum dolore eu ur. Excepteur sint occaecat&
\linenumannotation{6}cupidatat non proident, sunt in culpa qui officia deserunt\edlabel{end:2}&
\linenumannotation{7}\edlabel{begin:3}\edtext{Duis}{\xxref{begin:3}{end:3}\lemma{Duis\ldots occaecat}\Afootnote{Another note on two verses}} aute irure dolor in reprehenderit&
\linenumannotation{8}in voluptate velit esse cillum dolore eu fugiat nulla&
\linenumannotation{9}pariatur. Excepteur sint occaecat\edlabel{end:3}\&
\endnumbering

\end{document}
