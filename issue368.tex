% !TeX program = xelatex
% !TeX encoding = UTF-8
% !TeX spellcheck = it_IT

\documentclass[a4paper]{book}

\usepackage{fontspec}
\setmainfont{Linux Libertine O}

\usepackage{polyglossia}
\setmainlanguage[babelshorthands=true]{italian}

\usepackage[noledgroup,noeledsec,series={A,B,C}]{reledmac}      % EDIZIONE CRITICA
%\usepackage[noend,nofamiliar,noledgroup,series={A,B,C}]{reledmac}
\usepackage{reledpar}
\firstlinenum*{1}
\linenumincrement*{1}
\lineation{page}


\begin{document}
In this sample there are 8 scholia, but the first is transmitted in two different versions. None of the two versions must have a pstart number, since both are two types of scholion nr. 1.

On the contrary, line numbering should continue throughout the text. In this MWE, for example, there are 13 lines (from the scholion nr. 1).

\bigskip

\beginnumbering
\numberpstarttrue
\labelpstarttrue

\pstart\edlabel{test}%
    \edtext{}{\Bfootnote{{\textbf{\pstartref{test}}}\enspace nel mezzo del cammin}}%
    Scholion nr. 1%
\pend
\pausenumbering*

\begin{pairs}\sloppy
\begin{Leftside}
\resumenumbering*
    \pstart\noindent left side left side left side left side left side left side left side left side left side left side left side left side left side. \textbf{A}\pend
\pausenumbering*
\end{Leftside}
\begin{Rightside}
\resumenumbering*
    \pstart\noindent right side right side right side right side right side right side right side right side right side right side right side right side right side right side right side right side right side right side. \textbf{B}\pend
\pausenumbering*
\end{Rightside}
\end{pairs}
\Columns

\resumenumbering*

\pstart%
    Scholion nr. 2%
\pend

\pstart%
    Scholion nr. 3%
\pend

\pstart%
    Scholion nr. 4%
\pend

\pstart%
    Scholion nr. 5%
\pend

\pstart%
    Scholion nr. 6%
\pend

\pstart%
    Scholion nr. 7%
\pend

\pstart%
    Scholion nr. 8%
\pend

\labelpstartfalse
\endnumbering

\end{document}