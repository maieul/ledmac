\documentclass[10pt,twoside]{article}
\usepackage[series={A,B,C}]{reledmac}

\usepackage{polyglossia, xunicode}
\usepackage{libertine}
\setmainfont[Ligatures=TeX]{Linux Libertine}

\setmainlanguage{german}
\setotherlanguage{hebrew}






\usepackage{bidi}\autofootnoterule




\wrapcontentX[A]{\texthebrew}
\makeatletter
\renewcommand{\footnoteruleA}{\right@footnoterule}
\makeatother
\hangindentX{2em}
\def\speaker#1{#1}
\def\speakerd#1{#1}
\begin{document}



 \begin{hebrew}
 \setRTL
 \beginnumbering


\pstart {\RL{\speaker{חתן.}
{גם אַתה}.
 אייער פראַה און קינדער זעללע לעבע!}}
 \pend


\pstart {\RL{\speaker{יוקב.}
 מייא איהר לייט, מער זעללט נאך אַביסכה וואַרטע מיט דעם עססע איך ווילל ערשט
 דען שופט רופע לאָססע, דער זאָלל
אַהך מיט עססע. האַלט אַביסכה אין! שמואל, רוהף אמויהל דען
{צאָן}!\footnoteA[4]{\RL{צאָן ווירד געוואֶהנליך דורך שאַאף איבערזעטצט (נאַך מענדעלזאָהן: קליינעס פֿיה). שאַף אין
י\~{{\RL{}}}ודיש-דייטשער
 מונדאַרט.
 שויף. אֶהנעלנד מיט שופט.
 דאַהער וואַהרשיינליך דיע אַנאַלאָגישע בענעננונג דער שולצען בייא דען לאנדיודען. צאָן. איים פלוראַל אַבער צאָננע. צאָננעכער נעננען זיע דען שאַאפֿהירטען.
}}
}}\pend



     \endnumbering
\end{hebrew}

\newpage


\beginnumbering



\pstart \speakerd{Bräutigam.} Auch du, Eure Frau und Kinder sollen leben! \pend

\pstart \speakerd{Jakob.} Mei ihr Leute, man sollte noch ein bisschen warten mit dem Essen, ich will erst den Schultheiß rufen lassen, der soll auch mitessen. Haltet ein bisschen ein! Samuel, ruf einmal den \textit{Son} [Schultheiß].\footnoteB[4]{\textit{Son} wird gewöhnlich durch Schaf übersetzt (nach Mendelsohn: kleines Fieh). Schaf in jüdisch-deutscher Mundart: \textit{Schauf}/\textit{Schoif} ähnelnd mit \textit{Schuft}. Daher wahrscheinlich die analogische Benennung der Schulzen bei den Landjuden. \textit{Son}. Im Plural aber \textit{Sonne}. \textit{Sonnecher} nennen sie den Schafhirten.} \pend
\endnumbering


\end{document}
