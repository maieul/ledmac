
\documentclass[11pt]{article}
\title{Turkish Embassy Letters}
\author{ed. Rebecca Chung}  
\usepackage[parapparatus]{eledmac}
\usepackage{times, ulem, verse, setspace}
\usepackage[protrusion=true,expansion=true]{microtype}  
\renewcommand{\rmdefault}{ptm} 

\newcommand{\A}{\Afootnote}
\newcommand{\B}{\Bfootnote}
\newcommand{\edt}{\edtext}
\newcommand{\I}{\textit}
\newcommand{\U}{\uline}
\newcommand{\OED}{(\I{OED})}   
\newcommand{\CL}{\I{CL}}
\newcommand{\CT}{(comma tail removed) H}
\newcommand{\PE}{(paragraph ends) }
 \newcommand{\D}{\ldots}                                    

\setcounter{page}{8}
\setcounter{firstlinenum}{0}
\setlength\parindent{11pt}
\renewcommand*{\thefootnoteA}{\alph{footnoteA}}
\renewcommand{\notenumfont}{\bfseries\footnotesize}
\renewcommand*{\thefootnoteB}{\alph{footnoteB}}
\renewcommand{\notenumfont}{\bfseries\footnotesize}
\footparagraph{A}
\foottwocol{A}
\footparagraphX{B}
\frenchspacing
\noendnotes
\maxhXnotes{1\textheight}
\begin{document}
\beginnumbering 

\pstart

\begin{center}
\edt{Letter 1}{\A{1 H; LETTER I (Roman numerals throughout) F}}
\linebreak
To the \edt{Countess of ------}{\B{Montagu's younger sister, Lady Frances Pierrepont (1690--1761) married John Erskine (1675--1732) 6th Earl of Mar in 1714.   One year later, Mar led the failed 1715 Jacobite rebellion, which tried to depose George I (1660--1727), and restore exiled prince  James (Jacobus) Stuart (1688--1766).  Mar fled to France and was trying to raise another army.  Lady Mar's interests were protected by Montagu (1689--1762) and her husband Edward Wortley Montagu (1678--1761), as well as by father Evelyn Pierrepont (1655--1726), Duke of Kingston-upon-Hull, Lord Privy Seal (keeper of the royal seal for personal signatures).  By 1728, Lady Mar was mentally ill.  Montagu addresses Lady Mar in Letters 1, 7, 9, 14, 15, 16, 19, 21, 23, 29, 33, 39, 45, and 46; see also Letter 50, l.7n. and 13n., for excerpts from Lady Mar's own letters to her husband.}}\raisebox{-1pt}{.}\linebreak\linebreak
\end{center}
\pend
\pstart
\begin{flushright}
\edt{\edt{\I{Rotterdam}}{\B{a prosperous Dutch port.}}, Friday, \edt{Aug. 3}{\B{Plans kept changing.  \I{The Flying Post or The Post Master}, 1716 Apr. 14: ``Mr. Wortley Montague, late one of the Lords of the Treasury, is going Ambassador to Constantinople, in the Room of Colonel Sutton, who has obtain'd Leave to come Home."  Jul. 3, office circular: ``Mr. Montague, being appointed to succeed Sir Robert Sutton as his Majesty's ambassador at the Ottoman Port, a man-of-war is ordered to convey him to Constantinople" (Polwarth, 1:36).  Jul. 7: ``The honourable Edward Wortley, alias Montague, will set forward this Week on his Embassy to Constantinople"; ``Edward Wortley, alias Montague, Esq; Ambassador Extraordinary to the Ottoman Porte, sets out in a few Days for Constantinople, to relieve Sir Robert Sutton" (\I{Shift Shifted}; \I{Weekly Journal or British Gazetteer}). Jul. 27, John Robethon (d. 1722, secretary to George I) to Alexander Hume Campbell, Lord Polwarth (1675--1740, ambassador-extraordinary to Copenhagen): ``The Danish ambassador left in one of the King's yachts for Holland on his way to Hanover this morning, and at the same time Mr. Wortley Montagu left in another yacht also for Holland, whence he proceeds by land to Constantinople" (\I{Polwarth}, 1:47).  Jul. 28: ``Edward-Wortley Montague, Esq; set out Yesterday Morning on his Embassy to Constantinople" (\I{Weekly Packet}).  Jul. 31, office circular enclosure: ``Mr. Wortley Montague, his Majesty's ambassador extraordinary to Constantinople, sets out to-morrow to begin his journey thither" (\I{Polwarth},1:49).  Aug. 3, office circular: ``Last Wednesday being the anniversary of his Majesty's happy accession to the throne \ldots.The same day Mr. Wortley Montagu set out for his embassy \ldots " (\I{Polwarth}, 1:52).  A man-of-war was the largest and most-heavily armed sailing ship in the British navy; for yacht, see l.8n.}}. \edt{O.S.}{\B{To convert from old-style British dates to new-style (European) Continental dates,  add eleven days to the British date.  Where needed for quick referencing, both dates are supplied [using brackets].}} 1716.}{\lemma{\I{Rotterdam}\D1716.}\A{Rotterdam, Friday Aug{\textsuperscript{t}} 3. O.S. H}}\end{flushright}
\pend

\pstart
\hspace{-11pt}I flatter myself (dear sister) that I shall give you some pleasure in letting you know that I \edt{am}{\A{have F}} safely passed the sea, though we had the ill fortune of a storm.  
\pend
\pstart
We were persuaded by the captain of \edt{our}{\A{the F; our P}} \edt{yacht}{\A{yacht 1751; Yatcht HP; yatcht F}\B{In Dutch, \I{Jacht} means \I{hunt}; the Dutch used the light, fast-sailing yacht to combat pirates.  In England, the yacht became a leisure ship after Charles II (1630--1685) returned from exile in one.}} to set out in a \edt{calm}{\B{absence of wind.}}, and he pretended there was nothing so easy as to tide it \edt{over;}{\A{over, L4, B}} \edt{but}{\A{but, F}} after two days slowly moving, the wind blew so \edt{hard}{\A{hard, F}} that none of the sailors could keep their feet, and we were all Sunday night tossed very handsomely.  \pend
\pstart I never saw a man more \edt{frighted}{\A{frightened B}} than the \edt{captain.  For}{\A{Captain, for H; captain.  For F \D .L4}} my \edt{part}{\A{part, FB}} I have been so \edt{lucky}{\A{lucky, F}} neither to suffer from fear nor \edt{\edt{\edt{seasickness}{\B{An \I{ECCO} search (1700--1800) gives sixty-three results for \I{seasickness}.  Sixty-one results show line-end hyphenation: \I{sea- / sickness}.  One breaks the line: \I{sea sickness}.}},}{\A{seasickness 1765; sea sickness H; sea-sickness; FB}} \edt{though} {\A{th\^{o} H; tho', F}} I \edt{confess}{\A{confess, F}} I was so impatient to see myself once more upon dry land,}{\lemma{seasickness\D land}\B{Poet and friend Alexander Pope (1688--1744): ``Your first short letter only serves to show me you are living; It puts me in mind of the first Dove that return'd to Noah, \& just made him know it had found no Rest abroad.  There is nothing in it that can please me, but when you say you had no Seasickness" (1716 Aug. 20; \I{Corr.}, 1:356).    The two had become friends sometime in 1714.  Pope, who was working on his translation of Homer's \I{Iliad} (1714--1720), counted Montagu's father among his subscribers: ``I also got 2 Guineas from the Marquess of Dorchester" (To Jervas, 1715 Jun. 12, \I{Corr}. 1:296).  In winter 1716, Pope had helped Wortley become ambassador by arranging a meeting with former  ambassador to the Ottoman empire and France, William Trumbull (1639--1716); at the time, Montagu was recovering from court scandal and smallpox (Letter 25, l.8; Letter 31, l.31--32n.).  Pope's first surviving letter to Montagu is short, and is dated just prior to her departure: ``Madam,------So natural as I find it is to me, to neglect every body else in your company, I am sensible I ought to do any thing that might please you; and I fancy'd, upon recollection, our writing the Letter you proposed [a joint letter to mutual friend Lady Rich; see Letter 10, l.2n.] was of that nature.  I therfore sate down to my part of it last Night, when I should have gone out of town.  Whether or not you will order me, in recompence, to see you again, I leave to you; for indeed I find I begin to behave my self worse to you than to any other Woman, as I value you more.  And yet if I thought I shou'd not see you again, I would say some things here, which I could not to your Person.  For I would not have you dye deceivd in me, that is, go to Constantinople without knowing, that I am to some degree of Extravagance, as well as with the utmost Reason, Madam Your most faithfull \& most obedient humble Servant A Pope" (1716 Jul., \I{Corr}, 1:345).  Pope thought Montagu would travel by sea: ``This letter is written on the 20th of August tho it will scarce reach you in a month, at my Lord James Hay's Arrival at Leghorne [Ligorno, Italy].  I shall then be in a particular manner sollicitous for you, on your going again by Sea; and therfore beg the earliest notice of your safe Landing, on the other side" (James Hay unidentified; \I{Corr}, 1:358).  Pope and Montagu were friends c. 1715--1729.  Montagu addresses Pope in Letters 8, 22, 24, 30, 36, 48, and 52; she addresses Lady Rich in Letters 10, 18, 20, 26, 37, and 49.  Material from the Montagu-Pope correspondence appears throughout the \I{Letters}, but Montagu sometimes reassigns topics to others.}} that I would not stay till the yacht could get to \edt{Rotterdam}{\A{(italicized) F}}, but went in the \edt{longboat}{\A{longboat 1759; long boat H; long-boat F}\B{largest boat on the yacht.}} to \edt{Helvoet\-sluys}{\A{helver Sluyse, H; \I{Helvoetsluys} F; Helver Sluyse L4}}{\B{Hellevoetsluis, the Dutch war fleet home port.}}, \edt{where we}{\A{where \^{\textsuperscript{we}} P}} \edt{hired}{\A{hir'd H; had F}} \edt{voitures}{\B{carriages.}} to carry us to the \edt{\edt{Briel}{\B{Dutch seaport town.}}.}{\A{Brill, H; \I{Briel}. F; Brill. L4 B}}\pend

\endnumbering
\end{document}  
