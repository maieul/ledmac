\documentclass{book}

\usepackage{reledmac}
\setSErefprefixsingle{ligne~}
\setSErefprefixmore{lignes~}
\setSErefonlypageprefixsingle{p.~}
\setSErefonlypageprefixmore{pp.~}

\begin{document}

\SEref[prefixmore]{test}, \SEref[noprefix]{test2}

\beginnumbering
\autopar

\edlabelSE{test}Lorem ipsum dolor sit amet, consectetur sed do eiusmod tempor incididunt ut \edlabelSE{test2}labore et dolore magna amet aliqua.

\endnumbering

\end{document}
