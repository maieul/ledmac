\documentclass{scrbook}

\usepackage[utf8]{inputenc}
\usepackage[T1]{fontenc}
\usepackage[german]{babel}

\usepackage{letltxmacro}
\usepackage[xindy]{indextools}%
\indexsetup{level=\section*,toclevel=section}
\makeindex[title=Index generalis]
\makeindex[name=res,title=Index rerum]

\usepackage[xindy+hyperref]{reledmac}
\usepackage[hyperindex=false]{hyperref}

\makeatletter
\LetLtxMacro\orig@@index\index
\newcommandx\nindex[2][1,usedefault]{\orig@@index[#1]{#2|innote}}
\newcommand\innote[1]{#1\textit{n}}

\AtBeginDocument{%
\pretocmd{\@footnotetext}{\let\index\nindex}{}{}
\pretocmd{\footnoteA}{\let\index\nindex}{}{}
\apptocmd{\footnoteA}{\let\index\orig@@index}{}{}
\apptocmd{\@footnotetext}{\let\index\orig@@index}{}{}
}
\makeatother

\begin{document}

Textus\index{textus} usitatus\footnote{nota communis\index[res]{nota communis}}.

\begin{ledgroup}
\beginnumbering
\pstart
Textus\index{textus} a reledmac\footnoteA{Nota familiaris reledmacensis\index[res]{nota reledmacensis}} numeratus.
\pend
\endnumbering
\end{ledgroup}
\printindex
\printindex[res]

\end{document}