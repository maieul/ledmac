\documentclass{scrbook}

\usepackage[utf8]{inputenc}
\usepackage[T1]{fontenc}
\usepackage[german]{babel}

\usepackage{letltxmacro}
\usepackage[xindy,splitindex]{indextools}%
\indexsetup{level=\section*,toclevel=section}

\usepackage[series={A},noeledsec,noend,xindy,xindy+hyperref]{reledmac}
\usepackage[hyperindex=false]{hyperref}

%\newcommand{\fn}[1]{#1\textit{n}}

\makeatletter
\LetLtxMacro\orig@@index\index
\newcommand\nindex[1]{\orig@@index{#1|innote}}
\newcommand\innote[1]{#1\textit{n}}

\makeindex[name=res,title=Index rerum]
\makeindex[name=nom,title=Index nominum]

\AtBeginDocument{%
\pretocmd{\@footnotetext}{\let\index\nindex}{}{}
}
\makeatother


\begin{document}

Textus usitatus\footnote{nota communis\nindex[res]{nota communis}}.

\begin{ledgroup}
\beginnumbering
\pstart
Textus\index[res]{textus} a reledmac\footnoteA{Nota familiaris reledmacensis\nindex[nom]{reledmac}} numeratus.
\pend
\endnumbering
\end{ledgroup}

\printindex[res]
\printindex[nom]

\end{document}
