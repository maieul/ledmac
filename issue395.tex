\documentclass{book}
\usepackage[a4paper]{geometry}
\usepackage{fontspec}
\setmainfont[Ligatures=TeX]{Linux Libertine O}
\usepackage{xunicode,metalogo,hyperref}
\usepackage{polyglossia}

\usepackage[series={},nocritical,noend,nofamiliar,noledgroup]{reledmac}
\usepackage{reledpar} 

\linenumincrement*{1}
\firstlinenum*{1}
\setlength{\Lcolwidth}{0.42\textwidth}
\setlength{\Rcolwidth}{0.42\textwidth} 

\setmainlanguage{english}
\setotherlanguage{russian}
\setotherlanguage{hebrew}    
\setotherlanguage{arabic}    
\newfontfamily\arabicfont[Script=Arabic]{Scheherazade}
\newfontfamily\hebrewfont[Script=Hebrew]{Linux Libertine O}
\newfontfamily\russianfont[Script=Cyrillic]{Linux Libertine O}    

\begin{document}


\begin{pairs}

\begin{Rightside} 
\begin{RTL}
\begin{Arabic}
\beginnumbering
\pstart
\eledchapter{كلمة}\ledleftnote{s}
\pend
\pstart
كلمة كلمة كلمة كلمة كلمة كلمة كلمة كلمة 
كلمة كلمة كلمة كلمة كلمة كلمة كلمة كلمة كلمة كلمة كلمة كلمة كلمة كلمة 
كلمة كلمة كلمة كلمة كلمة كلمة كلمة كلمة كلمة كلمة كلمة كلمة كلمة كلمة كلمة كلمة كلمة كلمة كلمة كلمة ،كلمة كلمة كلمة كلمة كلمة  كلمة كلمة كلمة كلمة كلمة  
\pend    
\endnumbering
\end{Arabic}
\end{RTL}
\end{Rightside}

\begin{Leftside} 
\begin{russian}
\beginnumbering
\pstart
\eledchapter{Трактат Второй}   
\pend 
\pstart
О краеугольных [принципах] Торы, имеется ввиду, которые [есть] основы и столпы на которых дом Б-жий опирается/нахон, и с существованием их может быть представлено существование Торы упорядоченной от Него, благословенного, и если бы было представлено отсутствие одного из них — упадет Тора в общем, [Б-же] упаси.
\pend
\endnumbering
\end{russian}
\end{Leftside}

\end{pairs}
\Columns
\end{document}
