\documentclass{book}
\usepackage[nomath,osf]{libertinus-otf}
\usepackage[babelshorthands=true]{polyglossia}
	\setdefaultlanguage[variant=ancient]{greek}
	\setotherlanguage[variant=modern]{latin}
	\setotherlanguage[]{german}
	\setotherlanguage[]{italian}
  \usepackage[noledgroup,noeledsec,series={},nofamiliar,nocritical,noend]{reledmac}
\usepackage[continuousnumberingwithcolumns]{reledpar}
%\usepackage[style=italian]{csquotes}

\begin{document}
\textitalian{Quel ramo del lago di Como, che volge a mezzogiorno, tra due catene non interrotte di monti, tutto a seni e a golfi, a seconda dello sporgere e del rientrare di quelli, vien, quasi a un tratto, a ristringersi, e a prender corso e figura di fiume, tra un promontorio a destra, e un’ampia costiera dall’altra parte; e il ponte, che ivi congiunge le due rive, par che renda ancor più sensibile all’occhio questa trasformazione, e segni il punto in cui il lago cessa, e l’Adda rincomincia, per ripigliar poi nome di lago dove le rive, allontanandosi di nuovo, lascian l’acqua distendersi e rallentarsi in nuovi golfi e in nuovi seni. La costiera, formata dal deposito di tre grossi torrenti, scende appoggiata a due monti contigui, l’uno detto di san Martino, l’altro, con voce lombarda, il Resegone, dai molti suoi cocuzzoli in fila, che in vero lo fanno somigliare a una sega: talchè non è chi, al primo vederlo, purchè sia di fronte, come per esempio di su le mura di Milano che guardano a settentrione, non lo discerna tosto, a un tal contrassegno, in quella lunga e vasta giogaia, dagli altri monti di nome più oscuro e di forma più comune. Per un buon pezzo, la costa sale con un pendìo lento e continuo; poi si rompe in poggi e in valloncelli, in erte e in ispianate, secondo l’ossatura de’ due monti, e il lavoro dell’acque.}

Ὅτι μὲν ὑμεῖς, ὦ ἄνδρες Ἀθηναῖοι, πεπόνθατε ὑπὸ τῶν ἐμῶν κατηγόρων, οὐκ οἶδα· ἐγὼ δ' οὖν καὶ αὐτὸς ὑπ' αὐτῶν ὀλίγου ἐμαυτοῦ ἐπελαθόμην, οὕτω πιθανῶς ἔλεγον. καίτοι ἀληθές γε ὡς ἔπος εἰπεῖν οὐδὲν εἰρήκασιν. μάλιστα δὲ αὐτῶν ἓν ἐθαύμασα τῶν πολλῶν ὧν ἐψεύσαντο, τοῦτο ἐν ᾧ ἔλεγον ὡς χρῆν ὑμᾶς εὐλαβεῖσθαι μὴ ὑπ' ἐμοῦ ἐξαπατηθῆτε ὡς δεινοῦ ὄντος λέγειν. τὸ γὰρ μὴ αἰσχυνθῆναι ὅτι αὐτίκα ὑπ' ἐμοῦ ἐξελεγχθήσονται ἔργῳ, ἐπειδὰν μηδ' ὁπωστιοῦν φαίνωμαι δεινὸς λέγειν, τοῦτό μοι ἔδοξεν αὐτῶν ἀναισχυντότατον εἶναι, εἰ μὴ ἄρα δεινὸν καλοῦσιν οὗτοι λέγειν τὸν τἀληθῆ λέγοντα
\end{document}
