% !TEX TS-program = xelatexmk
 \documentclass{memoir}
	
\usepackage{libertine}

\usepackage{filecontents}
    \begin{filecontents}{\jobname.bib}
    @inbook{Bouzon1995,
        Author = {E. Bouzon},
        Booksubtitle = {Mélanges von Soden},
        Booktitle = {Vom alten Orient zum alten Testament},
        Editor = {Manfried Dietrich and Oswald Loretz},
        Hyphenation = {german},
        Location = {Neukirchen},
        Number = {240},
        Pages = {11--30},
        Publisher = {Neukirchener Verlag},
        Series = {Alter Orient und altes Testament},
        Shorthand = {\textsc{aoat}~240:2},
        Shorthandsort = {aoat 240:2},
        Shorttitle = {\emph{ṣimdat-šarrim}},
        Title = {Die soziale Bedeutung des \emph{ṣimdat-šarrim}-Aktes nach den Kaufverträgen der Rim--Sin-Zeit},
        Volume = {2},
        Year = {1995}}
    }
    \end{filecontents}

\usepackage[%
	notes,compresspages,% 321--328 --> 321--28
	isbn=false,			% don't need that in the body of the work; bibliography ok
	backref=true,
	block=space,		% biblatex; add space between bibliography blocks
%	refsegment=chapter,	% biblatex; start new reference segment at specified division
	autopunct=true,		% biblatex; after citations scan ahead for punctuation
%	firstinits,
	backend=biber,
	language=british,
	hyperref=true
	]{biblatex-chicago}

	\addbibresource{\jobname.bib}

\usepackage{hyperref,morewrites}

\usepackage{reledmac}
	\Xarrangement[A]{paragraph}		% format all A-series critical notes into a single paragraph
	\Xlemmaseparator[A]{:}			% replace ] with : in critical notes
	\Xendlemmaseparator[A]{:}
	\setlength{\linenumsep}{0.5em}	% set space between line number and margin
	\sublinenumberstyle{alph}		% subline numbers 'a', 'b', etc.
	\Xendbhooknote{\vskip 6pt\footnoterule}
	\renewcommand{\printnpnum}[1]{}
	\Xendparagraph[A]
%	\Xendafternote[A]{1em plus.4em minus.4em}
	\Xendtxtbeforenotes{\vskip 6pt\footnoterule}
	\newcommand{\nochange}[1]{#1}
	\Xwraplinenumannotation{\nochange}
\usepackage{reledpar} 
	\setlength{\Lcolwidth}{0.49\textwidth}
	\setlength{\Rcolwidth}{0.49\textwidth} 
	\firstlinenum*{1}
	\linenumincrement*{1}
	\setstanzaindents{1,0}
	\setlength{\stanzaindentbase}{1em}
	\setcounter{stanzaindentsrepetition}{1}
	\AtBeginPairs{\sloppy}

\newcommand{\smn}[1]{{\texorpdfstring{\sffamily\addfontfeature{LetterSpace=10.0}#1}{#1}}}	% Sumerian
\newcommand{\akk}[1]{{\texorpdfstring{\sffamily\textit{#1}}{#1}}}							% Akkadian
\newcommand{\cunchar}[1]{{\texorpdfstring{\textsc{#1}}{#1}}}
\newcommand{\D}{{\sffamily\textsuperscript{d}}}
\newcommand{\dtm}[1]{{\sffamily\textsuperscript{#1}}}



\newcommand{\qm}{\textsuperscript{?}}
\newcommand{\xm}{\textsuperscript{!}}
\newcommand{\gap}{[\dots]}

\begin{document}
\appendix
\chapter{Text Editions}\label{app:editions}

Line of text

\begin{pairs}
\begin{Leftside}
\beginnumbering
\begin{astanza}
\skipnumbering Copy A (Case) obverse\&
\end{astanza}
\endnumbering
\end{Leftside}
\begin{Rightside}
\beginnumbering
\begin{astanza}
\skipnumbering Copy B (Tablet) obverse\&
\end{astanza}
\endnumbering
\end{Rightside}
\end{pairs}
\Columns

\begin{pairs}
\OnehalfSpacing
\begin{Leftside}
\beginnumbering
\begin{astanza}
\skipnumbering\&
\end{astanza}
\begin{astanza}
\skipnumbering\&
\end{astanza}
\begin{astanza}
\skipnumbering\&
\end{astanza}
\begin{astanza}
\skipnumbering[\dots]\&
\end{astanza}
\begin{astanza}
\linenumannotation{′}\smn{[ki] }\dtm{d+}\cunchar{en-zu}--\akk{iš-me-a-ni}\&
\end{astanza}
\begin{astanza}
\linenumannotation{′}\dtm{I}\akk{iš}₈+\akk{tár}--\akk{ì-lí}\&
\end{astanza}
\begin{astanza}
\linenumannotation{′}\smn{in-ši-sa₁₀}\&
\end{astanza}
\endnumbering
\end{Leftside}
\begin{Rightside}
\beginnumbering
\begin{astanza}
\smn{1 sar igi-6-gál é-kislah}\&
\end{astanza}
\begin{astanza}
\smn{da é }\dtm{d+}\cunchar{en-zu}--\akk{be-el}--\akk{ì-lí}\&
\end{astanza}
\begin{astanza}
\smn{ù }\dtm{d+}\cunchar{en-zu}--\akk{še-mi}\&
\end{astanza}
\begin{astanza}
\smn{é }\dtm{d+}\cunchar{en-zu}--\akk{iš-me-a-ni}\&
\end{astanza}
\begin{astanza}
\smn{ ki }\dtm{d+}\cunchar{en-zu}--\akk{iš-me-‹a›-ni}\smn{\ dumu }\dtm{d+}\cunchar{en-zu}--\akk{a-bi}\&
\end{astanza}
\begin{astanza}
\dtm{I}\akk{iš}₈+\akk{tár}--\smn{dingir}\&
\end{astanza}
\begin{astanza}
\smn{in-ši-sa₁₀}\&
\end{astanza}
\endnumbering
\end{Rightside}
\end{pairs}
\Columns


(\textcite{MSL:1})


\begin{pairs}
\begin{Leftside}
\beginnumbering
\begin{astanza}
\skipnumbering {VS:13} (Sippar)\&
\end{astanza}
\endnumbering
\end{Leftside}
\begin{Rightside}
\beginnumbering
\begin{astanza}
\skipnumbering {VS:13} (`Larsa')\&
\end{astanza}
\endnumbering
\end{Rightside}
\end{pairs}
\Columns

\begin{pairs}
\OnehalfSpacing
\begin{Leftside}
\beginnumbering
\setline{12}
\begin{astanza}
\smn{u₄-kúr-še tukumbi}&
\smn{é }\akk{ba-aq-ri ir-ta-ši}&
\akk{a-lum ù ši-bu-tum}&
\akk{a-lum ù ši-bu-tum}&
\akk{i-ta-na-ap-pa-lu}\&
\end{astanza}
\endnumbering
\end{Leftside}
\begin{Rightside}
\beginnumbering
\begin{astanza}
\smn{[u₄-kúr]-še }\dtm{giš}\smn{kiri₆ inim-gál-la}&
\smn{ba-an-tuk}&
\smn{inim-gál-la }\dtm{giš}\smn{kiri₆-ke₄}&
\smn{ba-ni-ib-gi₄-gi₄}\&
\end{astanza}
\endnumbering
\end{Rightside}
\end{pairs}
\Columns

\end{document}
