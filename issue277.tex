\listfiles
\documentclass{book}
\usepackage[german]{babel} 
\usepackage[T1]{fontenc}
\usepackage[utf8]{inputenc}
\usepackage{blindtext}
\usepackage[series={A}]{eledmac}
\usepackage{eledpar}
\begin{document}
\noeledsec
Einleitende Worte \dots
%\newpage
\begin{pages}
  \begin{Leftside}
    \beginnumbering
    \pstart[%
      \section{Das erste Kapitel}%
    \subsection{Der erste Abschnitt}]

    Der Originaltext: \blindtext[2] Ja, das ist er!
    \pend

    \pstart[\section{Der zweite Abschnitt}]

    Der Originaltext: \blindtext[3] Ja, das ist er!
    \pend
    \endnumbering
  \end{Leftside}

  \begin{Rightside}
    \beginnumbering
    \pstart[%
      \section{Das erste Kapitel}%
    \subsection{Der erste Abschnitt}]

    Der übersetzte Text, das ist er! \blindtext[2] Und eine Zeile mehr, die ein bißchen Unruhe in den Parallelsatz bringt.

    \pend

    \pstart[\section{Der zweite Abschnitt}]

    Der \edtext{übersetzte Text}{\lemma{Übersetzter Text}\Aendnote{Kommentar: \blindtext[2]}}: das ist er! \blindtext[3] Mit ein bißchen bla, bla, bla, man weiß schon, wofür \dots
    \pend

    \endnumbering
  \end{Rightside}
\end{pages}
\Pages

\end{document}
