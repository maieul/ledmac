%Quand on met des annotations aux numéros de ligne au "leftside", cela cause des annotations de numéros de ligne dans l'appareil critique appartenant au "rightside" bien qu'il ne devrait pas. Au MWE on voit que le deuxième "endnote", donc celui pour le texte du "rightside", porte la numérotation "1^AR" au lieu de "1R".
% !TEX TS-program = xelatexmk
\documentclass{memoir}

\usepackage{reledmac}

\usepackage{reledpar} 
	\setstanzaindents{1,0}
	\setcounter{stanzaindentsrepetition}{1}
  \firstlinenum*{1}
  \linenumincrement*{1}
\begin{document}

\begin{pairs}
    \begin{Leftside}
        \beginnumbering
            \begin{astanza}
                \linenumannotation{A}First \edtext{left}{\Aendnote{<left endnote>}} paragraph.&
                Second left paragraph.\&
            \end{astanza}
        \endnumbering
    \end{Leftside}
    \begin{Rightside}
        \beginnumbering
        \autopar
        Only \edtext{right}{\Aendnote{<right endnote>}} paragraph.\pend
        \endnumbering
    \end{Rightside}

\end{pairs} 
\Columns
\doendnotesbysection{A}

\end{document}
