\documentclass[11pt,twoside]{book}

\usepackage{fontspec}
\usepackage{xunicode}
\usepackage[nomath]{libertinus}
\usepackage[parapparatus,draft]{reledmac}
\usepackage[shiftedpstarts]{reledpar}
\setlength{\columnrulewidth}{0.4pt}
\Xarrangement[A]{paragraph}
\Xarrangement[B]{paragraph}
\Xarrangement[C]{paragraph}
\Xarrangement[D]{paragraph}

\sublinenumberstyle{alph}
\let\fullstop\relax

\newcommand{\sktt}{\setlength{\leftskip}{15mm}\setlength{\rightskip}{15mm}\itshape}
\newcommand{\skte}{\setlength{\leftskip}{15mm}\setlength{\rightskip}{15mm}\itshape\skipnumbering}


\setlength{\parindent}{0pt}
\setcounter{stanzaindentsrepetition}{0}
\numberstanzafalse

\AtEveryPend{\ifinstanza\else\vspace{\parskip}\fi}

\Xstanza
\Xstanzaseparator{.}
\setRlineflag{}

\Xpstart

\hidenumberingonrightpage
\sethangingsymbol{[\,}
\linenumincrement{1}
\firstlinenumR{100000}
\firstlinenum{1}
\firstsublinenum{1}
\sublinenumincrement{1}
\setstanzaindents{1,0,0}
\setcounter{stanzaindentsrepetition}{1}
\Xnolinenumberifannotation[A]
\Xwraplinenumannotation{}

\newcounter{mydivision}

\Xtxtbeforenumber{{\themydivision}.}
\newcommand{\newdiv}{\refstepcounter{mydivision}}
\newcommand{\pstartdiv}{\pstart[\newdiv]\setline{1}\ledsidenote{\ \textbf{\themydivision}}}

\begin{document}

\begin{pages}
\begin{Leftside}
\beginnumbering


\skipnumbering
\pstart
\centering{||} O{||} \textit{avighnam astu}{||} O{||}
\pend

\pstartdiv
\noindent{}\edtext{saṅ}{\Bfootnote{em.; sa LOr}} saṅgrahakāri sira movus{|} liṅnira{|}
\pend

\begin{astanza}
\skipnumbering\linenumannotation{a}\itshape\edtext{śūnyaś}{\Afootnote{em.; śūnya A}}  ca \edtext{nirbbāṇādhikaḥ{|} \linenumannotation{b}śivāṅgatve}{\Afootnote{em; nirbbhāṇādhika{|} śśivaṅgatve A, B}} nirīkṣyate{|}&
\skipnumbering\linenumannotation{c}\itshape kutaḥ tadvākyam atulaṁ{|} \linenumannotation{d}śrutvā devo \edtext{’vatiṣṭhati}{\Afootnote{em; vatiṣṭha{|} ca A, B}} {||} \normalshape 1\&[ ]
\end{astanza}


\pstart
nāhan takvanaknaniṅ hulun ri bhaṭāra{|} hana ya pada śūnya{|} ya sinaṅguh ka\-mo\-kṣan{|} ṅa{|} viśeṣa ya{|} ya śiva ṅaranya{|} \textit{nirīkṣyate}{|} katon pva ya de saṅ \edtext{yogīśvara}{\lemma{\ldots{}gīśvara}\Cfootnote{B om.}}{|} sājñā bhaṭāra{|} an maṅkana kottamaniṅ vuvus saṅ ṛṣi{|} ya ta kaṛṅĕ de bhaṭāra{||}
\pend

\pstartdiv
\linenumannotation{a}\edtext{\textit{\edtext{kenopāyenax bhagavanx}{\Afootnote{em.; meyopāyena bhagavān LOr}\Dfootnote{unmetrical (\textit{bha-vipulā} with short second syllable)}}}{|}\linenumannotation{b}\textit{sukhayogasya lakṣaṇaṁ}}{\lemma{kenopāyena \ldots\ lakṣaṇam}\Cfootnote{cf. BhK 25.1}}{|} \linenumannotation{}\edtext{nihanxy}{\Bfootnote{conj.; nāhan A}} takvananiṅ hulun ri kita bha\-ṭā\-ra{|} ndya kunaṅ \edtext{upāyanikā}{\Bfootnote{corr.; opāyanikā A}} saṅ paṇḍita{|} mataṅyan kapaṅguhāvaknikaṅ sukhādhyātmika{||}
\pend

\pstartdiv
\textit{deva uvāca}{|} \edtext{mojar ta}{\Bfootnote{mojar PDok}} bhaṭāra{|} liṅnira{||}
\pend

\begin{astanza}
\skipnumbering\itshape
\linenumannotation{a}svaśarīre mahāyogī{|} \linenumannotation{b}\edtext{paśyate}{\Dfootnote{for dr̥śyate?}} hr̥dayāntare{|}&
\skipnumbering\linenumannotation{c}\itshape\edtext{vākyan}{\Dfootnote{vācyan ?}}  te parameśānam{|} \linenumannotation{d}\edtext{sūryyāyutasamaprabham}{\Afootnote{em.; °tamamapra° all mss.}}{||} \normalshape 3\&[ ]
\end{astanza}


\pstart
nihan \edtext{vuvusaniṅ}{\Bfootnote{vuvusniṅ PDok}} hulun i kita{|} ikā saṅ mahāyogī{|} sira tumon bhaṭāra \edtext{parameśvara}{\Bfootnote{prameśvara PDok}}{|} sateja sira lāvan tejaniṅ āditya sāyuta{|} \edtext{ṅkā ri śarīranira}{\Bfootnote{ṅkā riṅ śarīraniraṅ PDok}}  mvaṅ ri hatinira{||}
\pend

\pstartdiv
\linenumannotation{a}\edtext{\textit{\edtext{kenopāyena bhagavan}{\Afootnote{em.; meyopāyena bhagavān LOr}\Dfootnote{unmetrical (\textit{bha-vipulā} with short second syllable)}}}{|}\linenumannotation{b}\textit{sukhayogasya lakṣaṇaṁ}}{\lemma{kenopāyena \ldots\ lakṣaṇam}\Cfootnote{cf. BhK 25.1}}{|} \linenumannotation{}\edtext{nihanx}{\Bfootnote{conj.; nāhan A}} takvananiṅ hulun ri kita bha\-ṭā\-ra{|} ndya kunaṅ \edtext{upāyanikā}{\Bfootnote{corr.; opāyanikā A}} saṅ paṇḍita{|} mataṅyan kapaṅguhāvaknikaṅ sukhādhyātmika{|} \linenumannotation{c}another Sanskrit half-line{|} old javanese prose{||}
\pend

\pstartdiv
\linenumannotation{a}\edtext{\textit{\edtext{kenopāyena bhagavan}{\Afootnote{em.; meyopāyena bhagavān LOr}\Dfootnote{unmetrical (\textit{bha-vipulā} with short second syllable)}}} \linenumannotation{b}{|}\textit{sukhayogasya lakṣaṇaṁ}}{\lemma{kenopāyena \ldots\ lakṣaṇam}\Cfootnote{cf. BhK 25.1}}{|} \linenumannotation{c}\edtext{some more Sanskrit some more Sanskrit spanning two lines}{\Afootnote{test}}\linenumannotation{}\edtext{nihanx}{\Bfootnote{conj.; nāhan A}} takvananiṅ hulun ri kita bha\-ṭā\-ra{|} ndya kunaṅ \edtext{upāyanikā}{\Bfootnote{corr.; opāyanikā A}} saṅ paṇḍita{|} mataṅyan kapaṅguhāvaknikaṅ sukhādhyātmika{||} 5
\pend

\endnumbering

\end{Leftside}

    \begin{Rightside}
    \beginnumbering

{\centering\pstart Everything works here \pend}

\autopar

Everything works here:

\begin{astanza}
\setlength{\rightskip}{15mm}\itshape Everything works here. Everything works here. Everything works here. Everything works here. Everything works here. 1\&[ ]
\end{astanza}


Everything works here. Everything works here. Everything works here. Everything works here. Everything works here. Everything works here. Everything works here. Everything works here. Everything works here. Everything works here. (1)\



\textit{Everything works here. Everything works here}. Everything works here. Everything works here. Everything works here. Everything works here. Everything works here. (2)

Everything works here.

\begin{astanza}
\setlength{\rightskip}{15mm}\itshape Everything works here. Everything works here. Everything works here. Everything works here. Everything works here. 3\&[ ]
\end{astanza}

Everything works here. Everything works here. Everything works here. Everything works here. Everything works here. Everything works here. Everything works here (3)

Everything works here. Everything works here. Everything works here. Everything works here. Everything works here. Everything works here. Everything works here. Everything works here. Everything works here. Everything works here. (4)\



Everything works here. Everything works here. Everything works here. Everything works here. Everything works here. Everything works here. Everything works here. Everything works here. Everything works here. Everything works here. (5)\



    \endnumbering
        \end{Rightside}

\end{pages}
\Pages

\end{document}
