\documentclass[twoside,12pt]{book}
\usepackage{fontspec,xunicode,polyglossia}
\usepackage{libertineotf}
\usepackage[series={A,B,C,D},noend,noledgroup]{reledmac}
\usepackage[advancedshiftedpstarts]{reledpar}

\setgoalfraction{0.6}

% languages configuration
\setmainlanguage{italian}
\setotherlanguage{latin}

\Xonlyside[A]{L}
\Xonlyside[B]{L}
\Xonlyside[C]{R}
\Xonlyside[D]{L}

\begin{document}
	
\begin{pages}
	\begin{Leftside}
		\beginnumbering
		\pstart
		\skipnumbering\eledsection{Rusticum alloquitur}
		\pend
		\setcounter{stanzaindentsrepetition}{2}
		\setstanzaindents{8,0,1}
		
		\begin{astanza}[\\]
			Rustice, \edtext{neglecti cultor male fortis agelli}{\Cfootnote{Ovid. \textit{Fas}. 5, 499}},&
			Qui subigis pigro \edtext{rura paterna bove}{\Cfootnote{Hor. \textit{Epod}. 2, 1-3}},&
			Dumosoque sinis obduci limite campum,&
			Duraque non certo subruis arva die,&
			\edtext{Fallacisque tibi Cereris}{\Cfootnote{Hor. \textit{Carm}. 3, 1, 30 sgg.; Hor. \textit{Epod}. 16, 43 sgg.; Ovid. \textit{Ars} 1, 399; Verg. \textit{Aen}. 6, 343-344}}, fallacis et uvae&
			Fructus, et ingrati poma minuta soli,&
			Non seges haec, non cura levis, non lenta colendi&
			Desidia, offensi sed facit \edtext{ira dei}{\Cfootnote{Verg. \textit{Georg}. 4, 453-454; Ovid. \textit{Fas}. 1, 481 sgg.; Tib. 3, 6, 25-26}};\&
		\end{astanza}
		\begin{astanza}
			Cuius \edtext{numen aquae}{\Cfootnote{Ovid. \textit{Her.} 15, 157-158}} violas, dum saepe lutosam&
			A stabulis vitreo perluis amne suem,&
			Annua nec \edtext{veteri}{\Afootnote{veteri \textit{ex} veteris P}} reddis sua \edtext{vota sacello}{\Cfootnote{Prop. 2, 19, 13}},&
			Sed caedis \edtext{sacra relligione nemus}{\Cfootnote{Ovid. \textit{Fas}. 3, 264; Ovid. \textit{Am.} 3, 1, 1; Ovid. \textit{Fas}. 3, 295 sgg.}}.\&
		\end{astanza}
		\begin{astanza}
			\edtext{Ira deos tangit}{\Cfootnote{Ovid. \textit{Met}. 8, 279; Ovid. \textit{Fas}. 5, 297; Lucr. 2, 651}}. Coelestem vince furorem&
			Thure pio, et multam sedulus adde precem;\&
		\end{astanza}
		\begin{astanza}
			\edtext{Agna cadat Fauno}{\Cfootnote{Ovid. \textit{Fas}. 1, 720; Verg. \textit{Ecl}. 1, 7-8}}, \edtext{pinguisque ex hubere matris}{\Cfootnote{Verg. \textit{Georg}. 3, 187}}&
			Imbuat offensas \edtext{candidus hedus}{\Cfootnote{Tib. 2, 5, 38; Hor. \textit{Carm}. 3, 18, 5; Prop. 2, 19, 14}} aquas;&
			\edtext{Placabisque deum}{\Cfootnote{Verg. \textit{Georg}. 4, 547; Hor. \textit{Carm.} 1, 36, 1-3}}, sacrae qui praesidet undae,&
			\edtext{Viscera}{\lemma{viscera tosta}\Cfootnote{Verg. \textit{Aen}. 8, 180}} de niveo tosta ferens vitulo.\&
		\end{astanza}
		\endnumbering
		
	\end{Leftside}
	
	\begin{Rightside}
		\beginnumbering
		\pstart
		\skipnumbering\eledsection{Al contadino }
		\pend
		
		\pstart[\\]
		\noindent O contadino, poco energico coltivatore del trascurato campicello, tu che lavori i terreni paterni con un bue indolente e permetti che a cingere il campo ci siano i rovi, tu che rimandi ad un giorno non mai stabilito di zappare la terra indurita, i frutti di un’ingannevole Cerere e di un’uva ingannevole, i prodotti meschini di una terra ingrata, non questo campo, non il tuo poco lavoro, non la pigrizia e la flemma nel coltivare li han dati, ma l’ira di un dio che è stato offeso; 
		\pend
		\pstart
		tu ne oltraggi il potere che detiene sull’acqua, quando spesso dalle stalle porti la scrofa fangosa a lavarsi nel limpido fiume e non rendi all’antico tempietto le annuali e dovute offerte votive, ma di un bosco sacro fai legna da ardere. 
		\pend
		\pstart
		Gli dei sono soggetti all’ira. Vinci la collera divina con il pio incenso, aggiungendovi presto molte preghiere;
		\pend
		\pstart
		sia l’agnella sacrificata a Fauno e un bianco capretto, grasso per il nutrimento materno, plachi col sangue le acque oltraggiate; e placherai così anche il dio che presiede al sacro flutto, recandogli in dono le interiora bruciate di un vitello bianco. 
		\pend
		
		\endnumbering
	\end{Rightside}	
\end{pages}
\Pages 

	
\end{document}