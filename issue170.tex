\listfiles

\documentclass[a4paper,11pt]{report}

%% dimensions du livre

% \makeatletter
\hoffset -1in
\voffset -1in
\setlength{\topmargin}{45mm}
\setlength{\textwidth}{110mm}
\setlength{\textheight}{175mm}
\setlength{\oddsidemargin}{50mm}
\setlength{\evensidemargin}{50mm}
% \makeatother

\usepackage{xltxtra}

\setmainfont[Mapping=tex-text]{%
  FreeSerif%
}

\usepackage[sanskrit,frenchb]{babel}

\usepackage[parapparatus]{eledmac}
\let\variantes=\Afootnote
\Xnotenumfont{\bfseries}
\Xhangindent{14pt}

\catcode`\_=13 %
\newenvironment{velthuisrom}{%
  \addfontfeature{Mapping=velthuis-romanized}
  \catcode`\~=12 %
  \catcode`\_=13
  \def_{\space}
  \selectlanguage{sanskrit}%
  %% aparat critique
  \newcommand{\add}{\textit{add.\ }}%
  \newcommand{\om}{\textit{om.\ }}%
  \newcommand{\ill}{\textit{ill.\ }}%
}%
{\selectlanguage{frenchb}}
\catcode`\_=8

\newcounter{sutra}
\newcommand{\sutram}{|$\!$|\thesutra|$\!$|\par}

\makeatletter
\newenvironment{sutra}{\bfseries%
  \refstepcounter{sutra}
                \let\\\@centercr
                \list{}{\itemsep      \z@
                        \itemindent   -1.5em%
                        \listparindent\itemindent
                        \advance\leftmargin 1em}%
                \item\relax}
{\endlist}
\makeatother

\newcommand{\vers}{\hspace*{10mm}}

\newenvironment{demons}{%
  \parindent=0pt%
}{}%

\newcommand\kara[2]{%
\setlength{\arrayrulewidth}{.6pt}%
\begin{tabular}[t]{|c|}
 #1\\{}#2 \\\hline%
 \end{tabular}\kern 0pt
\vadjust{\vskip 3ex}}%

\newcommand\karan[2]{%
{\tiny%
\setlength{\arrayrulewidth}{.3pt}%
\begin{tabular}{|@{\rule{1.5pt}{0pt}}c@{\rule{1.5pt}{0pt}}|}%
 #1\\%
 #2\\\hline%
 \end{tabular}\kern 0pt%
}}%
\newcommand{\esp}{$\diamondsuit$\ }


\flushbottom


\begin{document}

\lineation{page} 



% \rule{1pt}{5cm}

% \chapter{Trairāśikam}

% \beginnumbering

% \pstart

% \ledchapter{Trairāśikam}


\begin{velthuisrom}
  \beginnumbering

  \pstart

  \edtext{%
    atha trairaa"sike kara.nasuutrav.rttam---}%
  {\linenum{||||\lineref{note1}}\lemma{atha... vilome}%
    \variantes{atha- \om -.R -kha \esp raa"sike- raa"sikena -aa \esp
      suutra- suutram -aa -kha \esp tat(1)- ta -aa| ta.h -i -ka \esp
      jaati- jaatii -aa| jaati.h -i \esp aadi(2)- aadya -aa \esp
      vidhi.h- dhi.r -.r}%
  }%

  \begin{sutra}
    {\bfseries%
      pramaa.nam_icchaa ca samaanajaatii\\
      \vers aadyantayos_tatphalam_anyajaati|\\
      madhye tad_icchaahatam_aadih.rt_syaad\\
      \vers icchaaphala.m vyastavidhir_vilome\edlabel{note1} \sutram}%
  \end{sutra}

  \edtext{%
    trayaa.naa.m raa"sii.naa.m samaahaaras_triraa"sika.m
    triraa"sikasyeda.m ga.nita.m trairaa"sikam| tadraa"sitraya
    aadyantayor_yau raa"sii tau samaanajaaty_ekajaatii bhavata eka.h
    pramaa.naraa"sir_apara icchaaraa"sir_iti|
    tayor_madhye_.anyajaatis_tatphala.m phalaraa"sis_tasya
    pramaa.nasya phalam| athavaa tatphala.m
    pramaa.naphalam_ity_artha.h| tatphalam_icchaahatam_aadyena
    pramaa.nena h.rtam_icchaaphala.m bhavati|}%
  {\linenum{||||\lineref{note2}}\lemma{trayaa.naam... h.rtam_iti}%
    \variantes{ga.nitam- \om -u -.r \esp trairaa"sikam- \om -i \esp
      raa"si(3)- raasi -ka| raa"sike -kha \esp aadyantayo.h- aadya
      antayo.h -i -uu \esp ekajaatii- \om -.r \esp eka.h- eka -aa -i
      -uu \esp iti(1)- \om -aa \esp tayo.h- tayo -.r \esp anya-
      anyatara -i \esp jaati.h- jaati -i -.R \esp tat(2)- etat -i -uu|
      tata.h -kha \esp phalam(1)- \om -aa -.R -ka -kha| jatam \add -u
      \esp phalam(2)- tatphalam \add -u \esp phalam(3)- pala -i| \om
      -ka \esp pramaa.na- \om -.r| pramaa.nam -.R -ka \esp
      phalam(4)...iti(3)- \om -.r \esp tat(3)- etat -i -uu \esp
      phalam(5)- \om -.R -ka \esp h.rtam(1)- hatam -aa -i -kha \esp
      ichaa(3)- bha \add -i \esp vilome- viloma -i| vilomai -ka \esp
      vyasta(1)- vidhi \add -aa -.R| \om -ka \esp raa"sike- raa"sika
      -aa -kha \esp vidhi.h- vidhe.h -i}%
  }%

  \edtext{%
    vilome vyastatrairaa"sike
    vyastavidhir_aadyahatam_icchaah.rtam_iti|\edlabel{note2}}%
  {}%

  \edtext{%
    atrodde"saka udaahara.nam---}%
  {\linenum{||||\lineref{note3}}\lemma{atra... kiyat}%
    \variantes{%
      atra...udaahara.nam- idam_udde"sakenoddi"sati -aa \esp atra-
      \om -.r \esp ud\-de"s\-aka- udde"sakam -i -u -.r \esp udaahara.nam-
      udaa@ -u \esp pala- pada -i \esp sapadi- sa yadi -aa \esp tat-
      \om -aa}%
  }

  \begin{sutra}
    {\bfseries%
      ku"nkumasya sadala.m paladvaya.m\\
      \vers ni.skasaptamalavais_tribhir_yadi|\\
      praapyate sapadi me va.nigvara\\
      \vers bruuhi ni.skanavakena tatkiyat\edlabel{note3}
      \sutram}
  \end{sutra}%

  \edtext{%
    ku"nkumasya saardha.m paladvaya.m ni.skasya tribhi.h
    saptaa.m"sair_yadi labhyate taddhe va.nigvara ni.skanavakena
    kiyat_ku"nkuma.m praapyate sapadi bruuhiiti|}%
  {\lemma{ku"nkumasya... bruuhiiti}%
    \variantes{%
      sa- \om -aa -i \esp ardham- \om -i \esp sapta(1)- saptama -u -ka
      -kha \esp ku"nkumam- ku"nkumasya -i -uu \esp praapyate- tad \add
      -.r| labhyate -.R \esp sapadi- sa yadi -aa \esp iti- \om -u
      -.r}%
  }%

  \edtext{%
    atra samaanajaatii ni.skaraa"sii tatra
    pramaa.namuulyatvaat_saptaa.m_"satraya.m pramaa.naraa"si.h|
    ni.skanavakam_icchaaraa"si.h|
    ku"nkumaraa"sir_anyajaatitvaat_pramaa.naphalatvaac_ca
    madhyaraa"sir_iti|} %
  {\lemma{atra... raa"sir_iti}%
    \variantes{%
      jaatii- iti \add -i \esp raa"sii- raa"si -.R \esp tatra- \om -aa
      \esp pramaa.na(1)- pramaa.nam -i -uu| pramaa.naa -.r \esp
      trayam- yam -i \esp icchaa- iccha -.R \esp anya- a -.r \esp
      jaati- \om -u \esp tvaat(2)- tad \add -aa| tvaa -.R| k.rta \add
      -.R}%
  }%

  \edtext{%
    nyaasa.h
    \begin{tabular}[t]{@{}|c|c|c|@{}}
      3 & 5 & 9\\
      7 & 2 & 1\\\hline
    \end{tabular}}{}%

  \begin{demons}
    \edtext{icchaahata.m madhyam \kara{45}{2}}%
    {\linenum{||||\lineref{note4}}\lemma{iccha... kar.sau 2}%
      \variantes{%
        icchaa- yaantena \karan{9}{1} madhya.m \karan{5}{2} \add -u
        \esp hatam(1)- haram -kha \esp viparyaa- viparyayaa -i|
        viparyayaa"sena -u|viparyayaase -.r \esp sa.mk.rtvaa-
        sa.mk.rtyaa -aa| \om -u| k.rtvaa -.r| sa.mk.rtvaat -ka \esp
        hatam(2)- h.r -.R \esp chedena- cheda -u -kha \esp h.rtam-
        h.rte -.r| k.rtam -.R \esp ku"nkuma- ku"nkumaani -aa \esp 3-
        \om -aa \esp pala- \om -.r \esp bhaagatvaat- bhaagak.rtvaat -i
        \esp kar.sasya- kar.sakasya -.R \esp gu.nam- 12 \add -ka \esp
        chedena(2)- chedam -u \esp puna.h- la -ka}%
    }%

    aadyena cchedaa.m"saviparyaasa.m k.rtvaa hata.m \kara{315}{6}\,|

    chedena h.rta.m labdhaani ku"nkumapalaani 52 "se.sa.m 3|

    palacaturthabhaagatvaatkar.sasyeti caturgu.na.m chedena
    punarbhakta.m labdhau kar.sau 2|\edlabel{note4}
  \end{demons}

\pend

\endnumbering

\end{velthuisrom}

% \pend

% \endnumbering

\end{document}

