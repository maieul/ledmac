
\listfiles
\documentclass{book}
\usepackage[german]{babel} 
\usepackage[T1]{fontenc}
\usepackage[utf8]{inputenc}
\usepackage{blindtext}
\usepackage[series={},noeledsec,noledgroup]{eledmac}
\usepackage{eledpar}

\clubpenalty = 10000
\widowpenalty = 10000


\begin{document}

\numberpstarttrue

\begin{pages}
\begin{Leftside}
\beginnumbering
\pstart
\blindtext
%the following line is empty – or not ...

\pend

\pstart
\blindtext
\pend
\endnumbering   
\end{Leftside}

\begin{Rightside}
\beginnumbering
\pstart
Dies hier ist ein Blindtext zum Testen von Textausgaben. Wer diesen Text liest,
ist selbst schuld. Der Text gibt lediglich den Grauwert der Schrift an. Ist das wirk-
lich so? Ist es gleichgültig, ob ich schreibe: „Dies ist ein Blindtext“ oder „Huardest
gefburn“? Kjift – mitnichten! Ein Blindtext bietet mir wichtige Informationen. An
ihm messe ich die Lesbarkeit einer Schrift, ihre Anmutung, wie harmonisch die Fi-
guren zueinander stehen und prüfe, wie breit oder schmal sie läuft. Ein Blindtext
sollte möglichst viele verschiedene Buchstaben enthalten und in der Originalsprache
gesetzt sein.
\pend

\pstart
\blindtext
\pend

\endnumbering
\end{Rightside}
\end{pages}
\Pages
\end{document}
